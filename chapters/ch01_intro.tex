I set up a thesis latex project where I briefly describe the
      scope of the project. The project can be best described as quasi-static
      manipulation of elastic rods. There is a subtle difference between elastic
      rods and deformable linear objects (DLO's). Elastic rods can be described as a
      subset of DLO's and exhibit deformation and elasticity in fewer degrees of
      freedom. For my thesis, I will focus on solving elastic rod manipulation
      as it would be sufficient to resolve many existing bottlenecks in the
      industry. The thesis will be divided into five major sub-systems where
      each could be comprised of smaller research projects and papers.
      The five main systems are perception, tracking, chaos estimator,
      control and planning, and manipulation.
      The perception module is responsible for receiving, preprocessing and
      fusing raw data from RGB and depth sensors to provide rich, concise and
      persistent features to tracking and chaos estimator modules.
      The tracking module uses observation metadata such as the pose and velocity
      of each segment of the elastic rod. Moreover, it deploys a EKF estimator
      with a known dynamic model to accurately track the object in space. Chaos
      estimator module acts as a central and responsive node where it selectively
      picks what observations to tune into which ones to react to. Its output
      has priority over the control and planning module and it outputs directly
      to manipulation module.
      My current research project aims to start with a simple double pendulum simulation in python and gradually extend its degrees of freedom.
      Moreover, I will spring terms to the dynamic equations to simulate elasticity. I am currently working on a 2D simulation of a double pendulum and
      I am almost done. I am having trouble saving the animation but I should be able to resolve it very soon. A double pendulum can be extended and held
      stationary at the other end to form a quasi-static configuration. The more
      'particles' or 'nodes' that are added to the model the smoother the elasticity
      of the object becomes. Additionally, to simulate kirchhoff's elastic rod,
      we need to add a degree of freedom on the roll axes to each edge with its
      own rotational spring force. In a 3D space and with the elastic rod placed
      on a planar workbench, we form the \(\mathbb{R}^6\) configuration space
      mentioned in Dr. Bretl's paper that can express quasi-static configurations
      sufficiently, \cite{Quasi-static}.
