\documentclass[11pt]{article}
\usepackage{bookmark}
\usepackage{algorithm}
\usepackage{algpseudocode}
\usepackage{amsfonts}
\usepackage{amsmath}
\usepackage{amssymb}
\usepackage{amsthm}
\usepackage{bm}
\usepackage{color}
\usepackage{comment}
\usepackage{float}
\usepackage{graphicx}
%\usepackage[hidelinks]{hyperref}
\usepackage{makecell}
\usepackage[caption=false,font=footnotesize,subrefformat=parens,labelformat=parens]{subfig}
\usepackage{wrapfig}
\usepackage{url}
\usepackage[table]{xcolor}
%
\setlength{\parindent}{0.25in}
\setlength{\parskip}{.05in}
\pagestyle{plain}
%Title, date an author of the document
\title{Progress Report}
\author{Bardia Mojra}


\begin{document}
\maketitle
\thispagestyle{empty}

\bigskip
\bigskip
\begin{center}
      Robotic Vision Lab
\end{center}

\begin{center}
      The University of Texas at Arlington
\end{center}

\newpage

\section{Specific Research Goals}
\begin{itemize}
      \item Pose Estimation: Implement and improve.
      \item NBV-Grasping.
      \item Pose estimation survey.
      \item Universal pose estimation.
\end{itemize}

\section{To Do}
\begin{itemize}
      \item Catch up with my reading list.
      \item OpenCV: Get comfortable using it.
      \item NBV-Grasping: Update URDF file, add camera. Write two inverse kinematic
      function for the gripper and the camera.
      \item PVNet implementation: Paused. Working on a simple pose estimation for now.
      \item MSI Fellowship: On pause. Deadline was too close. Will read NASA papers
      and write up something for next opportunity.
      \item Look into methods of generating uncertainty data.
\end{itemize}

\section{Reading List}
\begin{itemize}
      \item VEst \cite{5160698}
      \item NASA papers \cite{NASATech44:online}
      \item Leveraging feature uncertainty in the pnp problem \cite{ferraz2014leveraging}
      \item Vision-based robotic grasping from object localization, object pose
      estimation to grasp estimation for parallel grippers - a review \cite{du2020vision}
      \item Berk Calli's YCB \cite{calli2015ycb}
      \item Normalized objects \cite{Wang_2019_CVPR}
      \item Roadmap \cite{roadmap251:online}

\end{itemize}

\section{Progress}
The following items are listed in the order of priority:
\begin{itemize}
      \item Pose Estimation: I am working on extracting key points and features
      from YCB dataset. I also looked into camera calibration. I had some difficulties
      but Joe and Chris recommended using a ROS package so I will pursue that.

      \item OpenCV: I have dome some basic tutorials. I need to become intimate
      with this.

      \item QuEst \cite{QuEst}: In this paper, authers cleverly use Quaternion notation
      to derive and develop a 5-point pose estimation scheme that is more robust to noise.
      Quaternion notation increases dimensions of the problem but rather preserve
      more numerical precision in computation. This become apparent as QuEst
      performs better that SOTA for image sequences under noisy conditions.
      Moreover, this method computes translation and rotation independently which
      also eliminates error propagation from translation estimation to rotation
      estimation. For similar reasons, they based their algorithm on a 5-point
      or more pose estimation algorithm which is explained in more detail in
      \cite{ma2012invitation}.

      \item NASA MSI Fellowship: Next, I will read papers from NASA
      \cite{NASATech44:online} and develop a proposal.

      \item PyTorch Tutorials: Transfer learning: I did this tutorial on PyTorch
      but I think I was too tired when I did that last week. I will do it again.

      \item PVNet: Next: Use transfer learning and ResNet to train a model for
      semantic segmentation on YCB dataset. -- I was working on this when over
      write my boot section. I welcome problems. I was copying YCB dataset from
      my internal backup drive to my partion when I used rsync without a
      destination. Rsync is powerfull command that allows for files to be copied
      using the entire communication bus bandwidth. I started writing a shell
      script to automate the recovery procedure.

      \item NBV Grasping Project: Joe and I setup ROS Client on UR5e and were
      able to control it via my workstation. We had to downgrade to Ubuntu 18
      because ROS Client driver uses ROS Melodic. After discussing this Chris,
      he recommended we move to Noetic and build the driver from source because
      ROS communication messages should match between Melodic and Noetic. I will
      test this on my computer first then implement it on my workstation. I don't
      want to undo Joe's work without a robust solution.

      \item UTARI: I found the source code for QuEst. I will convert it from
      Matlab code to Python and will play with it.

      \item Implement features from PoseCNN, DOPE, and BayesOD. - On pause.
\end{itemize}

%\newpage

\section{Immediate Plans - Summer 2021:}
The following items are listed in the order of priority:

\begin{itemize}
      \item Pose estimation: First, I will implement a simple pose estimation
      model and gradually will add feature extraction and other techniques for
      robust and fast pose estimation. I will have to learn how to extract
      features from ground truth data, i.g. label, bounding box, center, position,
      orientation and more. Some of these features are given but some need to
      calculated. So far, I have become familiar with Python development
      environment and 2D data manipulation. Next, I need to seek guidance on
      how to process 3D data sets. Coupled with what I have learned in CSE-6363
      and PyTorch tutorials, I am confident I can quickly develop a simple
      pose estimation model and improve it over the summer. I want to start
      writing a paper on this topic but it is difficult to set a timeline
      without a working implementation. Right after finals, I will resume working
      on this and read paper from CVPR and ICRA on the topic.

      \item NBV-Grasping: I will follow up with Chris and Joe and will try to
      assist and learn as much as I can. The goal is to write the paper by mid
      to end of the summer.
      \item UTARI: It depends on Dr. Gans' plan for the summer. Most likely, I will
      be working on phased array radar project.
\end{itemize}

\section{Intermediate Goals - Fall 2021:}
\begin{itemize}
      \item Pose estimation: I must be finished with implementation, perhaps
      make some improvements, and should be working on a paper for ICRA or CVPR.
      \item Scene understanding and active learning: After pose estimation, I
      want to expand my research into scene understanding and active learning in
      the context of advanced manifacturing.
      \item ARIAC: Once I am up to speed, I will do the ARIAC workshops/tutorials
      and will talk to Jerry about possible contributions.
\end{itemize}


%Sets the bibliography style to UNSRT and import the
\newpage
\bibliography{references}
\bibliographystyle{ieeetr}

\end{document}
