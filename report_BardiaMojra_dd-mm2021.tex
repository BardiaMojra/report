\documentclass[11pt]{article}
\usepackage{bookmark}
\usepackage{algorithm}
\usepackage{algpseudocode}
\usepackage{amsfonts}
\usepackage{amsmath}
\usepackage{amssymb}
\usepackage{amsthm}
\usepackage{bm}
\usepackage{color}
\usepackage{comment}
\usepackage{float}
\usepackage{graphicx}
%\usepackage[hidelinks]{hyperref}
\usepackage{makecell}
\usepackage[caption=false,font=footnotesize,subrefformat=parens,labelformat=parens]{subfig}
\usepackage{wrapfig}
\usepackage{url}
\usepackage[table]{xcolor}
%
\setlength{\parindent}{0.25in}
\setlength{\parskip}{.05in}
\pagestyle{plain}
%Title, date an author of the document
\title{Progress Report}
\author{Bardia Mojra}


\begin{document}
\maketitle
\thispagestyle{empty}

\bigskip
\bigskip
\begin{center}
      Robotic Vision Lab
\end{center}

\begin{center}
      The University of Texas at Arlington
\end{center}

\newpage

\section{Specific Research Goals}
\begin{itemize}
      \item Grant Proposal (Oct 14th): Keep working on this.
      \item VPQEKF (IROS - Mar. 1st): Work on the paper, focus on this in October.
      \item NBV-Grasping (IROS - Mar. 1st): Work on tasks assigned by Chris, one day a week. Focus on this from November till March.
\end{itemize}

\section{To Do}
\begin{itemize}
  \item Grant Proposal: Find three scientific questions to be researched. - Done
  \item Grant Proposal: Finish the first complete draft of the proposal over the weekend.
  \item Grant Proposal: Next week, start working on the personal statement, use UTA SOP as the initial draft.
  \item PVQEKF:
  \begin{itemize}
      \item Go over code and write matrix equations.
      \item I will go over the paper once every morning and expand sections for 30 minutes to an hour.
      \item Double-check my data prep implementation. Use KITTI Python module.
      \item Test with Hilti dataset.
      \item Add L2-norm and L2 loss features.
      \item I need to separate the state observation and control input vectors from the
      \item Develop object tracking and robust-to-truncation feature.
      \item Get ROS environment up and running. I need to install Armadillo (C++) with a certain dependency configuration.
  \end{itemize}

  \item Real-time pose estimation demo.
  \item NBV-Grasping:
      \begin{itemize}
      \item Update URDF and Xacro files for UR5e to include a sensor, sensor mount (with offset), and the gripper. -- Next
      \item Add movement constraints for tables and scenes.
      \item Write two IK functions for gripper and sensor, one for each. It should plug-in with MoveIt configurator.
      \item Research and implement point-cloud data to training TensorFlow models.
      \item Learn and implement GraspIt package.
      \end{itemize}
\end{itemize}


\section{Progress}
The following items are listed in the order of priority:
\begin{itemize}
      \item Fellowship: I read more papers on deformable object shape and pose estimation, as
      well as handling and control. Robotics researchers at Google recently published a paper where they introduced DeformableRavens, \cite{seita2020learning}, an open-source simulated benchmark for 1D, 2D and 3D object manipulation. Although Google's publication is both notable and insufficient, it is worth discussing another paper, \cite{hashempour2020data}. The authors present an empirical data set for a dual-arm surgical suturing application using the Da Vinchi robot. I need to read the paper more in the detail, I am interested in their method and approach. I plan on keeping the proposal focused on deformable linear objects (1D) as well as 1) to develop an empirical data-set for shape deformity estimation, 2) to develop an empirical data-set for object state dynamic estimation through \textit{a series of well defined dynamic tests} in dual-arm configuration, 3) to investigate the best methods for learning latent space representation with loss functions that penalize for shape deformity and dynamics estimation errors.
      \item VPQEKF: No update. I haven't done much with Hilti dataset after downloading it.
      \item NBV Grasping Project: No updates.
      \item PyTorch Tutorials: Transfer learning.
      \item Pose Estimation: On pause.
      \item SD Team: No update.
      \item EE Autonobots: No update.
\end{itemize}

%\newpage

\section{Intermediate Goals - Fall 2021:}
\begin{itemize}
      \item QEKF: Finish paper.
      \item Active Learning.
      \item ARIAC: Once I am up to speed, I will do the ARIAC workshops/tutorials and will talk to Jerry about possible contributions.
\end{itemize}


%Sets the bibliography style to UNSRT and import the
\newpage
\bibliography{ref}
\bibliographystyle{ieeetr}

\end{document}
