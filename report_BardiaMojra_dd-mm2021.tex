\documentclass[11pt]{article}
\usepackage{bookmark}
\usepackage{algorithm}
\usepackage{algpseudocode}
\usepackage{amsfonts}
\usepackage{amsmath}
\usepackage{amssymb}
\usepackage{amsthm}
\usepackage{bm}
\usepackage{color}
\usepackage{comment}
\usepackage{float}
\usepackage{graphicx}
%\usepackage[hidelinks]{hyperref}
\usepackage{makecell}
\usepackage[caption=false,font=footnotesize,subrefformat=parens,labelformat=parens]{subfig}
\usepackage{wrapfig}
\usepackage{url}
\usepackage[table]{xcolor}
%
\setlength{\parindent}{0.25in}
\setlength{\parskip}{.05in}
\pagestyle{plain}
%Title, date an author of the document
\title{Progress Report}
\author{Bardia Mojra}


\begin{document}
\maketitle
\thispagestyle{empty}

\bigskip
\bigskip
\begin{center}
      Robotic Vision Lab
\end{center}

\begin{center}
      The University of Texas at Arlington
\end{center}

\newpage

\section{To Do}
\begin{itemize}
      \item PVNet implementation: Test and document, learn and rewrite.
      \item Implement pose estimation: Keypoint uncertainty, understand RANSAC.
      \item Look into methods of generating uncertainty data.
      \item Pose Estimation Servery: On pause.
      \item Vision-based robotic grasping from object localization, object pose
      estimation to grasp estimation for parallel grippers - a review,
      \cite{du2020vision}: Will read after PVNet implementation.
      \item Look into PyBullet for RL.
      \item Look into Facebook Flashlight C++ library, \cite{flashlig35:online}.
      \item Look into Nvidia Omniverse, \cite{NVIDIAOm1:online}.
\end{itemize}

\section{Reading List}
\begin{itemize}
      \item \cite{ferraz2014leveraging}
      \item \cite{he2015deep}
      \item \cite{du2020vision}
\end{itemize}

\section{Progress}
The following items are listed in the order of priority:
\begin{itemize}
      \item PyTorch Tutorials: I did more tutorials on custom neural network,
      CNN, image classification, and Tensorboard. Next, I will work on image
      segmentation and transfer learning tutorials.

      \item PVNet: I plan on recreating the algorithm with PyTorch and other
      available machine learning modules such as \cite{openmmla99:online}.

      \item NBV Grasping Project: I ended up converting STEP files to STL format
      and used SketchUp's web app to design a mount for RealSense L515. I sent
      the initial draft to Chris for review. I still need to add final touches
      for zipties to be placed flush to the edge of the mount. Next, I will do
      ROS-UR5 tutorials to prepare for this project.

      \item NASA MSI Fellowship: I plan on writing a proposal on robotic
      arm manipulation for multiple applications. I can incorporate ideas from pose
      estimation with uncertainty to make it an attractive proposal. The topic
      is of interest to NASA per their 2020 technology taxonomy plan, \cite{2020nasa40:online}. In the
      document, they refer to grappling technologies capable of capturing natural
      and man-made free flying objects. Moreover, they seem to be interested in
      dexterous manipulation where a robot can handle various types of object
      while achieving compliant force control for safe operation near human
      operators and in deep-space environments. These topics are well inline with
      our work at Robotic Vision Lab and hold great potential commercial and
      social impacts as such areas are encouraged by NASA MSI grant program.
      There is a general call for proposals in MSI grant solicitation for robotic
      technologies inline with NASA missions. I will write an initial draft,
      after some review, I can reach out to find a technical advisor.

      \item UTARI: No new development.

      \item YCB Dataset \cite{calli2015ycb}: Start with YCB data and look into
      Berk Calli's work.
      \item Normalized Objects \cite{Wang_2019_CVPR}:
      \item Implement features from PoseCNN, DOPE, and BayesOD. - On pause.
\end{itemize}

%\newpage

\section{Plans}
The following items are listed in the order of priority:

\begin{itemize}
      \item Pose Estimation in Simulation \cite{NVIDIAIs75:online}: Use Nvidia
      Isaac SDK for in-simulation pose estimation training.
      \item Look into domain randomization and adaptation techniques.
      \item Project Alpe with Nolan: On pause for right now.
      \item UR5e: Finish ROS Industrial tutorials.
\end{itemize}

\section{2021 Goals and Target Journals/Conferences}
\begin{itemize}
      \item Submit a paper on pose estimation with uncertainty to ICIRS.
      \item Get comfortable with TensorFlow and related Python modules.
      \item Keep writing.
\end{itemize}


%Sets the bibliography style to UNSRT and import the
\newpage
\bibliography{references}
\bibliographystyle{ieeetr}

\end{document}
