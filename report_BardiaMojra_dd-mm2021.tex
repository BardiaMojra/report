\documentclass[11pt]{article}
\usepackage{bookmark}
\usepackage{algorithm}
\usepackage{algpseudocode}
\usepackage{amsfonts}
\usepackage{amsmath}
\usepackage{amssymb}
\usepackage{amsthm}
\usepackage{bm}
\usepackage{color}
\usepackage{comment}
\usepackage{float}
\usepackage{graphicx}
%\usepackage[hidelinks]{hyperref}
\usepackage{makecell}
\usepackage[caption=false,font=footnotesize,subrefformat=parens,labelformat=parens]{subfig}
\usepackage{wrapfig}
\usepackage{url}
\usepackage[table]{xcolor}
%
\setlength{\parindent}{0.25in}
\setlength{\parskip}{.05in}
\pagestyle{plain}
%Title, date an author of the document
\title{Progress Report}
\author{Bardia Mojra}


\begin{document}
\maketitle
\thispagestyle{empty}

\bigskip
\bigskip
\begin{center}
      Robotic Vision Lab
\end{center}

\begin{center}
      The University of Texas at Arlington
\end{center}

\newpage

\section{Specific Research Goals}
\begin{itemize}
      \item VPQEKF (IROS - Mar. 1st): Work on the paper.
      \item DLO Manipulation Proposal: Work on a personal statement.
\end{itemize}

\section{To Do}
\begin{itemize}
  \item Fellowship - DLO:
  \begin{itemize}
      \item Unity dataset
      \item Real dataset
      \item Develop a well-written personal statement. --- On-going.
      \item Seek other graduate fellowship opportunities. --- On-going.
      \item Develop multiple versions of research and personal statements for
      submission to different opportunities.
  \end{itemize}
  \item PVQEKF (Paper deadline March 1st.):
  \begin{itemize}
      \item Setup ROS environment -- (1) -- due 12/7
      \item Restore github access
      \item Replace EKF with QEKF -- (2) -- due 12/7
      \item Feature point extraction:
      \item Depth to scale
      \item BigC (where we solve Q+V together) --> regarding depth scale issue
      \item Quat: switching problem is fixed
      \item 35 solutions (start here)
      \item Noise issue: noise cannot be modelled
      \item Chaining step: when feature points come in and out of the frame dependency configuration.
  \end{itemize}
\end{itemize}


\section{Progress}
The following items are listed in the order of priority:
\begin{itemize}
      \item Fellowship: No update.
      \item VPQEKF: I received a new dataset from Asif with acceleration features.
      I will have to write new data handler, update the dynamic model and QEKF
      materices. I will most likely have it done by next Thursday. Regarding
      implementation of QuEst and Vest \cite{quest}, it would be more convenient
      and faster if I rewrite the original Matlab implementation in Python. The
      current QEKF implementation is in Python as well.

      \item DLO: I wrote the initial draft for \emph{Stable Control of Double
      Inverted Pendulum via Feedback Linearization} paper. I derived the
      \emph{Equations of Motion} for the mentioned system and simulated it
      in Python. In order to prove the controller is \emph{Optimal}, I will have
      to derive the \emph{Lyapunov Control Function}, use \emph{Robust Control}
      formulation, and show it is \emph{stable in finite time}. I will probably
      not have the time to derive the \emph{Optimal Solution Proof} in full by
      end of the semester. Dr. Gans and I discussed the possibility of extending
      this work with addition of novel contributions and publishing it in a
      control systems conference. To finish the assignment, I will develop
      multiple test scenarios, run the experiment and discuss the results.
      Moreover, there will be a section dedicated to studying the \emph{Chaos
      Within} and show how quickly it appears in experiments.

      \item NBV-Grasping Project: No update.
      \item PyTorch Tutorials: Transfer learning.
      \item Pose Estimation: I will need it for DLO segment localization.
\end{itemize}

%\newpage

\section{Intermediate Goals - Fall 2021:}
\begin{itemize}
      \item QEKF: Finish paper.
      \item Active Learning.
      \item UR5e: Do the tutorials.
\end{itemize}


%Sets the bibliography style to UNSRT and import the
\newpage
\bibliography{ref}
\bibliographystyle{ieeetr}

\end{document}
