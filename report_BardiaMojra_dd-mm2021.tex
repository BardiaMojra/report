\documentclass[11pt]{article}
\usepackage{bookmark}
\usepackage{algorithm}
\usepackage{algpseudocode}
\usepackage{amsfonts}
\usepackage{amsmath}
\usepackage{amssymb}
\usepackage{amsthm}
\usepackage{bm}
\usepackage{color}
\usepackage{comment}
\usepackage{float}
\usepackage{graphicx}
%\usepackage[hidelinks]{hyperref}
\usepackage{makecell}
\usepackage[caption=false,font=footnotesize,subrefformat=parens,labelformat=parens]{subfig}
\usepackage{wrapfig}
\usepackage{url}
\usepackage[table]{xcolor}
%
\setlength{\parindent}{0.25in}
\setlength{\parskip}{.05in}
\pagestyle{plain}
%Title, date an author of the document
\title{Progress Report}
\author{Bardia Mojra}


\begin{document}
\maketitle
\thispagestyle{empty}

\bigskip
\bigskip
\begin{center}
      Robotic Vision Lab
\end{center}

\begin{center}
      The University of Texas at Arlington
\end{center}

\newpage

\section{To Do}
\begin{itemize}
      \item Implement pose estimation: Train OD on YCB data.
      \item Implement pose estimation: Train PE on YCB data.
      \item Implement DOPE with added dropout before each layer to estimate
            variational Bayesian inference.
      \item Data Association and Localization of Classified Objects in Visual
      SLAM \cite{iqbal2020data}: Finished reading and annotated a virtual copy.
      \item Pose Estimation Uncertainty paper: I wrote down some ideas
      and section drafts for the paper.
      \item Implement PoseCNN, DOPE, and BayesOD.
      \item Pose Estimation Servery: Will begin writing as soon as this weekend.
      \item Vision-based robotic grasping from object localization, object pose
      estimation to grasp estimation for parallel grippers - a review,
      \cite{du2020vision}:

\end{itemize}


\section{Progress}
The following items are listed in the order of priority:
\begin{itemize}
      \item Instead of writing a review on \cite{pinto2016curious}, I thought
      it would be better if I focus on my primary task and write down my ideas
      for pose estimation. I need to stay focused.
      \item Object Pose Estimation with Uncertainty: I finished setting up
      basic object detection, next I will train a
      model on YCB dataset. I went over PoseCNN and BayesOD code once again and
      it slightly more sense to me. At some point I want to add an object motion
      classification feature where
      the OD subsystem makes predictions about whether each object is
      stationary, moving passively, or moving actively. This object level
      classification feature will help with more advanced pose estimations by
      providing informative priors.

      I believe pose estimation becomes much easer task if there are known
      dimensions within the frame. Given known objects, or even some known
      objects in the image frame, one should be able to make accurate inference
      about most unknown linear dimensions in the image frame. To reduce
      computational load, I will train a network to detect keypoints for each
      segmented object and only perform pose estimation inference for those
      keypoints.

      Keypoint Pose Estimation:
      I believe overall object pose estimation could increase considerably with
      this proposed method, especially in occlusion cases. Once keypoints are
      detected, another network estimates xyz pose for each point with
      uncertainty which I will explain in sections bellow.

      Keypoint Pose Uncertainty:
      Once keypoints are segmented in each frame, a certainty region is then
      defined about that point in 3D. This region is divided into regions
      either as a hypercube or hypersphere. We do the same for all keypoints
      of all objects present in the scene and connect the keypoints for each
      object. This method eliminates the need to make a similar angular
      classification regions.

      For each frame, detected objects with high confidence are considered as
      acceptable predictions by the system and draw a 3D box containing all
      keypoints for each object. A network would be trained to predict a 3D bounding
      box given keypoint corners depth, dimensions of the known object, spatial
      relationship of high confidence objects (need to develop more) in form
      of embeddings. An IoU loss function \cite{yu2016unitbox} could be
      implemented to optimize 3D bounding boxes.

      This method mitigates uncertainty in pose estimation by only considering
      new evidence if the previous process made high confidence predictions and that
      new evidence should be \textit{consistent} with all previous evidence or
      this could increase uncertainty of the priors which is an inherent
      problem in all \textit{chaotic systems}. Robust
      control tackles this issue by assigning more weight to new evidence and
      slowly forgetting older evidence. Similar approaches in computer science
      have become more prominent in machine learning with the introduction
      transformers (long-short memory, I am not sure).

      Similar to weights of a neuron, certainty about a prior should decrease
      over time unless new supporting evidence is observed or it is classified
      as an anchor with lasting weight and high certainty. Anchors would be
      used to track keypoint from frame to frame. This brings me to embodiment
      and expansion into active learning where the agent interacts with the
      environment, seeks to reduce uncertainty and approaches similar to
      Next Best View (cite Chris's paper).

      \item YCB Dataset \cite{calli2015ycb}: Right now I am trying to figure
      out how to train a model using this dataset.
      \item Normalized Objects \cite{Wang_2019_CVPR}:
      \item Implement features from PoseCNN, DOPE, and BayesOD.
\end{itemize}

%\newpage

\section{Plans}
The following items are listed in the order of priority:

\begin{itemize}
      \item Pose Estimation in Simulation \cite{NVIDIAIs75:online}: Use Nvidia
      Isaac SDK for in-simulation pose estimation training.
      \item Look into domain randomization and adaptation techniques.
      \item Project Alpe with Nolan: On pause for right now.
      \item UR5e: Finish ROS Industrial tutorials.
\end{itemize}

\section{2021 Goals and Target Journals/Conferences}
\begin{itemize}
      \item Submit a paper on pose estimation with uncertainty to ICIRS.
      \item Get comfortable with TensorFlow and related Python modules.
      \item Keep writing.
\end{itemize}


%Sets the bibliography style to UNSRT and import the
\newpage
\bibliography{references}
\bibliographystyle{ieeetr}

\end{document}
