\documentclass[11pt]{article}
\usepackage{bookmark}
\usepackage{algorithm}
\usepackage{algpseudocode}
\usepackage{amsfonts}
\usepackage{amsmath}
\usepackage{amssymb}
\usepackage{amsthm}
\usepackage{bm}
\usepackage{color}
\usepackage{comment}
\usepackage{float}
\usepackage{graphicx}
%\usepackage[hidelinks]{hyperref}
\usepackage{makecell}
\usepackage[caption=false,font=footnotesize,subrefformat=parens,labelformat=parens]{subfig}
\usepackage{wrapfig}
\usepackage{url}
\usepackage[table]{xcolor}
%
\setlength{\parindent}{0.25in}
\setlength{\parskip}{.05in}
\pagestyle{plain}
%Title, date an author of the document
\title{Progress Report}
\author{Bardia Mojra}


\begin{document}
\maketitle
\thispagestyle{empty}

\bigskip
\bigskip
\begin{center}
      Robotic Vision Lab
\end{center}

\begin{center}
      The University of Texas at Arlington
\end{center}

\newpage

\section{Specific Research Goals}
\begin{itemize}
      \item VPQEKF (IROS - Mar. 1st): Work on the paper.
      \item DLO Manipulation Proposal: Work on a personal statement.
\end{itemize}

\section{To Do}
\begin{itemize}
  \item Fellowship:
  \begin{itemize}
      \item Develop a well-written personal statement. --- On-going.
      \item Seek other graduate fellowship opportunities. --- On-going.
      \item Develop multiple versions of research and personal statements for
      submission to different opportunities.
  \end{itemize}
  \item ICRA 2022 Paper Review: On-going.
  \item PVQEKF:
  \begin{itemize}
      \item Go over code and write matrix equations. --- Done.
      \item Write daily. --- On-going.
      \item Double-check my data prep implementation. Use KITTI Python module.
      \item Test with Hilti dataset.
      \item I need to separate the state observation and control input vectors from the z matrix. --- Done.
      \item Develop object tracking and robust-to-truncation feature.
      \item Get ROS environment up and running. I need to install Armadillo (C++) with a certain dependency configuration.
  \end{itemize}
\end{itemize}


\section{Progress}
The following items are listed in the order of priority:
\begin{itemize}
      \item Fellowship: I worked on my personal statement on two occasions. I will
      write an outline for the DLO manipulation project and break it down into smaller
      projects.
      \item VPQEKF: I discovered a new bug in the algorithm where it used ground truth
      rotation instead of the previous state's posterior belief. It seems to work fine now, there are L1 and L2 errors ranging from single digit to six digit values.

      \item ICRA 2022 - Pose Estimation with Double Quaternion Particle Filter:
      I have read through half of the paper and I am enjoying it very much. I
      familiarized myself with dual quaternion representation and I find it preferable
      over normal quaternion representation. Normal quaternion representation uses
      \textit{unit lines} or \textit{unit vectors} \textbf{i}, \textbf{j},
      \textbf{k} to represent a point (when \(w=0\)) or rotation vector (when \(||q||=1\)).
      Simply put, Clifford extended Hamilton's quaternion to dual quaternion by using
      a second quaternion to represented \textbf{ \(\Delta i\)},
      \textbf{\(\Delta j\)}, \textbf{\(\Delta k\)} with \(w=0\). Furthermore, this
      SO(6) representation is reduced SE(3) by defining \(\epsilon^2 = 0\) and using
      the Euler's screw axis. Dual quaternion simplifies pose and motion
      computation since it enables concatenation of translation and rotation
      matrices for batch computation while
      maintataining high numerical accuracy.

      \item NBV-Grasping Project: No update.
      \item PyTorch Tutorials: Transfer learning.
      \item Pose Estimation: I will need it for DLO segment localization.

\end{itemize}

%\newpage

\section{Intermediate Goals - Fall 2021:}
\begin{itemize}
      \item QEKF: Finish paper.
      \item Active Learning.
      \item UR5e: Do the tutorials.
\end{itemize}


%Sets the bibliography style to UNSRT and import the
\newpage
\bibliography{ref}
\bibliographystyle{ieeetr}

\end{document}
