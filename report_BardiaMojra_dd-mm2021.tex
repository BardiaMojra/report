\documentclass[11pt]{article}
\usepackage{bookmark}
\usepackage{algorithm}
\usepackage{algpseudocode}
\usepackage{amsfonts}
\usepackage{amsmath}
\usepackage{amssymb}
\usepackage{amsthm}
\usepackage{bm}
\usepackage{color}
\usepackage{comment}
\usepackage{float}
\usepackage{graphicx}
%\usepackage[hidelinks]{hyperref}
\usepackage{makecell}
\usepackage[caption=false,font=footnotesize,subrefformat=parens,labelformat=parens]{subfig}
\usepackage{wrapfig}
\usepackage{url}
\usepackage[table]{xcolor}
%
\setlength{\parindent}{0.25in}
\setlength{\parskip}{.05in}
\pagestyle{plain}
%Title, date an author of the document
\title{Progress Report}
\author{Bardia Mojra}


\begin{document}
\maketitle
\thispagestyle{empty}

\bigskip
\bigskip
\begin{center}
      Robotic Vision Lab
\end{center}

\begin{center}
      The University of Texas at Arlington
\end{center}

\newpage

\section{To Do}
\begin{itemize}
      \item PVNet implementation: Test and document, learn and rewrite.
      \item Implement pose estimation: Keypoint uncertainty, understand RANSAC.
      \item Look into methods of generating uncertainty data.
      \item Pose Estimation Servery: On pause.
      \item Vision-based robotic grasping from object localization, object pose
      estimation to grasp estimation for parallel grippers - a review,
      \cite{du2020vision}: Will read after PVNet implementation.
\end{itemize}

\section{Reading List}
\begin{itemize}
      \item \cite{ferraz2014leveraging}
      \item \cite{he2015deep}
      \item \cite{du2020vision}
\end{itemize}

\section{Progress}
The following items are listed in the order of priority:
\begin{itemize}
      \item Pose Estimation, PVNet \cite{peng2019pvnet}: After successfully
      setting up Cuda 9.0 and GCC 6.3 on my machine, I tried to compile PVNet's
      Cuda source code once again and it failed due to other deprecated
      dependencies. I
      finally reached out to Joe, explained the situation to him, and
      mentioned I had to disable Ubuntu's automatic package manager through
      universal repositories. He recommended using the Docker implementation as
      fixing the long chain of dependency issues is not worth the time or the
      effort.

      I followed the Docker implementation for PVNet-Clean, which is still based
      Cuda 9.0 (PVNet Docker implementation is updated to Cuda 10.2 but it is
      messier) and setup
      the environment successfully. I confirmed the setup by running PVNet
      Docker with the appropriate Nvidia container and compiled all five Cuda
      source code modules without any errors or significant warnings.

      Moreover, I downloaded the dataset and the pretrained models and setup them
      up to be used by the project. Next, I will perform some testing, document
      the results in a systematic fashion and then continue and propagate the
      same
      procedure throughout the entire code base. In the process, I will most
      likely come up with ideas for improvement and naturally begin a rewrite.
      Nevertheless, I should do a full rewrite because at this point, it will
      only make me implement and analyze source code faster.

      \item YCB Dataset \cite{calli2015ycb}: Start with YCB data and look into
      Berk Calli's work.
      \item Normalized Objects \cite{Wang_2019_CVPR}:
      \item Implement features from PoseCNN, DOPE, and BayesOD. - On pause.
\end{itemize}

%\newpage

\section{Plans}
The following items are listed in the order of priority:

\begin{itemize}
      \item Pose Estimation in Simulation \cite{NVIDIAIs75:online}: Use Nvidia
      Isaac SDK for in-simulation pose estimation training.
      \item Look into domain randomization and adaptation techniques.
      \item Project Alpe with Nolan: On pause for right now.
      \item UR5e: Finish ROS Industrial tutorials.
\end{itemize}

\section{2021 Goals and Target Journals/Conferences}
\begin{itemize}
      \item Submit a paper on pose estimation with uncertainty to ICIRS.
      \item Get comfortable with TensorFlow and related Python modules.
      \item Keep writing.
\end{itemize}


%Sets the bibliography style to UNSRT and import the
\newpage
\bibliography{references}
\bibliographystyle{ieeetr}

\end{document}
