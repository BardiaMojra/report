\documentclass[11pt]{article}
\usepackage{bookmark}
\usepackage{algorithm}
\usepackage{algpseudocode}
\usepackage{amsfonts}
\usepackage{amsmath}
\usepackage{amssymb}
\usepackage{amsthm}
\usepackage{bm}
\usepackage{color}
\usepackage{comment}
\usepackage{float}
\usepackage{graphicx}
%\usepackage[hidelinks]{hyperref}
\usepackage{makecell}
\usepackage[caption=false,font=footnotesize,subrefformat=parens,labelformat=parens]{subfig}
\usepackage{wrapfig}
\usepackage{url}
\usepackage[table]{xcolor}
%
\setlength{\parindent}{0.25in}
\setlength{\parskip}{.05in}
\pagestyle{plain}
%Title, date an author of the document
\title{Progress Report}
\author{Bardia Mojra}


\begin{document}
\maketitle
\thispagestyle{empty}

\bigskip
\bigskip
\begin{center}
      Robotic Vision Lab
\end{center}

\begin{center}
      The University of Texas at Arlington
\end{center}

\newpage

\section{To Do}
\begin{itemize}
      \item PVNet implementation: Debugging Cuda modules.
      \item Implement pose estimation: Keypoint uncertainty, understand RANSAC.
      \item Look into methods of generating uncertainty data.
      \item Pose Estimation Servery: On pause.
      \item Vision-based robotic grasping from object localization, object pose
      estimation to grasp estimation for parallel grippers - a review,
      \cite{du2020vision}: Will read after PVNet implementation.
\end{itemize}

\section{Reading List}
\begin{itemize}
      \item \cite{ferraz2014leveraging}
      \item \cite{he2015deep}
      \item \cite{du2020vision}
\end{itemize}

\section{Progress}
The following items are listed in the order of priority:
\begin{itemize}
      \item Pose Estimation: I am still working on PVNet Cuda modules, the code
      written in Cuda/C++ style, so I have been reading on that. Based on my
      understanding, I need to update Cuda and C++ API syntax. I learned about
      Clang, which is used for compiling and debugging C++ source code. I
      cloned, built and made the LLVM project, \cite{TheLLVMC47:online}, which
      took longer than I expected. It comes with an extensive test suite
      \cite{testsuit42:online} containing over
      thirdy two compilation tests and only a few failed which tells me I
      should be okay. I installed this because it is the recommended tool used
      with Visual Studio Code for debugging Cuda. CLang provides helpful
      warning and error messages and seems well documented.


      \item PVNet \cite{peng2019pvnet}:

      \item YCB Dataset \cite{calli2015ycb}: Start with YCB data and look into
      Berk Calli's work.
      \item Normalized Objects \cite{Wang_2019_CVPR}:
      \item Implement features from PoseCNN, DOPE, and BayesOD. - On pause.
\end{itemize}

%\newpage

\section{Plans}
The following items are listed in the order of priority:

\begin{itemize}
      \item Pose Estimation in Simulation \cite{NVIDIAIs75:online}: Use Nvidia
      Isaac SDK for in-simulation pose estimation training.
      \item Look into domain randomization and adaptation techniques.
      \item Project Alpe with Nolan: On pause for right now.
      \item UR5e: Finish ROS Industrial tutorials.
\end{itemize}

\section{2021 Goals and Target Journals/Conferences}
\begin{itemize}
      \item Submit a paper on pose estimation with uncertainty to ICIRS.
      \item Get comfortable with TensorFlow and related Python modules.
      \item Keep writing.
\end{itemize}


%Sets the bibliography style to UNSRT and import the
\newpage
\bibliography{references}
\bibliographystyle{ieeetr}

\end{document}
