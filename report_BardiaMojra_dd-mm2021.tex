\documentclass[11pt]{article}
\usepackage{bookmark}
\usepackage{algorithm}
\usepackage{algpseudocode}
\usepackage{amsfonts}
\usepackage{amsmath}
\usepackage{amssymb}
\usepackage{amsthm}
\usepackage{bm}
\usepackage{color}
\usepackage{comment}
\usepackage{float}
\usepackage{graphicx}
%\usepackage[hidelinks]{hyperref}
\usepackage{makecell}
\usepackage[caption=false,font=footnotesize,subrefformat=parens,labelformat=parens]{subfig}
\usepackage{wrapfig}
\usepackage{url}
\usepackage[table]{xcolor}
%
\setlength{\parindent}{0.25in}
\setlength{\parskip}{.05in}
\pagestyle{plain}
%Title, date an author of the document
\title{Progress Report}
\author{Bardia Mojra}


\begin{document}
\maketitle
\thispagestyle{empty}

\bigskip
\bigskip
\begin{center}
      Robotic Vision Lab
\end{center}

\begin{center}
      The University of Texas at Arlington
\end{center}

\newpage

\section{Specific Research Goals}
\begin{itemize}
      \item VPQEKF (IROS - Mar. 1st): Work on the paper.
      \item DLO Manipulation Proposal: Work on a personal statement.
\end{itemize}

\section{To Do}
\begin{itemize}
  \item Fellowship - DLO:
  \begin{itemize}
      \item Unity dataset
      \item Real dataset
      \item Develop a well-written personal statement. --- On-going.
      \item Seek other graduate fellowship opportunities. --- On-going.
      \item Develop multiple versions of research and personal statements for
      submission to different opportunities.
  \end{itemize}
  \item PVQEKF (Paper deadline March 1st.):
  \begin{itemize}
      \item Setup ROS environment -- (1) -- due 12/7
      \item Restore github access
      \item Replace EKF with QEKF -- (2) -- due 12/7
      \item Feature point extraction:
      \item Depth to scale
      \item BigC (where we solve Q+V together) --> regarding depth scale issue
      \item Quat: switching problem is fixed
      \item 35 solutions (start here)
      \item Noise issue: noise cannot be modelled
      \item Chaining step: when feature points come in and out of the frame dependency configuration.
  \end{itemize}
\end{itemize}


\section{Progress}
The following items are listed in the order of priority:
\begin{itemize}
      \item Fellowship: No update.
      \item VPQEKF: Dr. Gans, Cody, and I met and discussed QuEst+Vest
      side of the paper and went over the code. In the next two weeks and perhaps
      after finals, I will 1, resolve the pending issue with environment setup,
      and 2, replace the EKF implementation with my latest QEKF module.
      There is currently an issue with my setup or it could be due to a lack of
      proper installation documentation where I can not build the ROS package.
      This is the same
      issue where I got stuck and had to move on in the
      summer. I asked Jerry for help and he kindly tried with no success. During
      the meeting on Tuesday, I brought up the idea of rewriting the code and
      Dockerizing the ROS environment. We are considering omitting
      the ROS portion of the code in order to make the IROS deadline and
      to use timestamps instead.
      Cody and Dr. Gans
      seemed open-minded and somewhat intrigued with the idea of \emph{containers}
       as a technology.
      We discussed using Unity for deep learning data acquisition tasks \cite{quest}.


      \item DLO: I have been reading papers with derivation, simulation, and code
      on controlling double inverted pendulum. There are two main approaches
      to this problem, \emph{the LaGrange-Euler formulation} and \emph{Robust
      Lyapunov Control Function}. I successfully derived, simulated,
      and controlled a single inverted pendulum in simulation by deploying the
      \emph{LaGrange-Euler} approach. For the double inverted pendulum, I plan
      on using \emph{Robust Lyapunov Control Function}.

      \item NBV-Grasping Project: No update.
      \item PyTorch Tutorials: Transfer learning.
      \item Pose Estimation: I will need it for DLO segment localization.
\end{itemize}

%\newpage

\section{Intermediate Goals - Fall 2021:}
\begin{itemize}
      \item QEKF: Finish paper.
      \item Active Learning.
      \item UR5e: Do the tutorials.
\end{itemize}


%Sets the bibliography style to UNSRT and import the
\newpage
\bibliography{ref}
\bibliographystyle{ieeetr}

\end{document}
