\documentclass[11pt]{article}
\usepackage{bookmark}
\usepackage{algorithm}
\usepackage{algpseudocode}
\usepackage{amsfonts}
\usepackage{amsmath}
\usepackage{amssymb}
\usepackage{amsthm}
\usepackage{bm}
\usepackage{color}
\usepackage{comment}
\usepackage{float}
\usepackage{graphicx}
%\usepackage[hidelinks]{hyperref}
\usepackage{makecell}
\usepackage[caption=false,font=footnotesize,subrefformat=parens,labelformat=parens]{subfig}
\usepackage{wrapfig}
\usepackage{url}
\usepackage[table]{xcolor}
%
\setlength{\parindent}{0.25in}
\setlength{\parskip}{.05in}
\pagestyle{plain}
%Title, date an author of the document
\title{Progress Report}
\author{Bardia Mojra}


\begin{document}
\maketitle
\thispagestyle{empty}

\bigskip
\bigskip
\begin{center}
      Robotic Vision Lab
\end{center}

\begin{center}
      The University of Texas at Arlington
\end{center}

\newpage

\section{Specific Research Goals}
\begin{itemize}
      \item Pose Estimation: Implement and improve.
      \item NBV-Grasping.
      \item Pose estimation survey.
      \item Universal pose estimation.
\end{itemize}

\section{To Do}
\begin{itemize}
  \item Catch up on my reading list.
  \item Pose Estimation:
  \begin{itemize}
      \item Read VEst \cite{dani2009position}.
      \item Implement QuEst in Python.
      \item OpenCV: Get comfortable using it.
      \item Key point feature extraction.
      \item Evaluate various pose estimation methods: PnP, QuEst, and else.
      \item PVNet implementation: Paused. Working on a simple pose estimation for now.
      \item Look into ARKit \cite{Augmente8:online}.
      \item Look into ARCore \cite{Buildnew97:online}.
  \end{itemize}
  \item NBV-Grasping:
      \begin{itemize}
      \item Update URDF and Xacro files for UR5e to include sensor,
sensor mount (with offset), and the gripper.
      \item Add movement constrains for tables and scene.
      \item Write two IK functions for gripper and sensor, one for each. It
should plug-in with MoveIt configurator.
      \item Research and implement point-cloud data to training TensorFlow models.
      \item UR5e in simulation: Joe might consider.
      \item Learn and implement GraspIt package.
      \end{itemize}

  \item MSI Fellowship: On pause.
  \item Look into methods of generating uncertainty data.
\end{itemize}

\section{Reading List}
\begin{itemize}
      \item VEst \cite{dani2009position} - On-going
      \item Vision-based robotic grasping from object localization, object pose
      estimation to grasp estimation for parallel grippers - a review \cite{du2020vision}
      \item NASA papers \cite{NASATech44:online}
      \item Leveraging feature uncertainty in the pnp problem \cite{ferraz2014leveraging}
      \item Berk Calli's YCB \cite{calli2015ycb}
      \item Normalized objects \cite{Wang_2019_CVPR}
      \item Roadmap \cite{roadmap251:online}

\end{itemize}

\section{Progress}
The following items are listed in the order of priority:
\begin{itemize}
      \item Pose Estimation: I continued with my OpenCV tutorial. At this point,
      I am trying to learn how to extract key features from objects as the primary
      step in object pose estimation. My goal is to develop a robust pose estimator
      that is primarily based on classical approach and to leverage Neural Network
      where it is absolutely necessary or where its benefits outweight the training
      time, intraceability, and inference time costs.

      Moreover, I came across Objectron \cite{ahmadyan2020objectron} which is
      developed by a team at Google. The authors provide their precompiled library
      as part of MediaPipe \cite{lugaresi2019mediapipe} module. I tested the module
      and was able to detect bounding box of a cup using webcam (live feed)
      with about roughly 30 frames per second being processed. The performance
      is good as it works in a plug-and-play fashion but it is jumpy and could
      use some anchoring to stabilize the predicted bounding box from frame to
      frame. The authors
      contribute the fast inference time to available AR tool kits, e.g. ARKit
      \cite{Augmente8:online} and
      ARCore \cite{Buildnew97:online}, that provide sparse point clouds in 3D.
      They provide their data set containing 4 million densely annotated
      images and nearly 15 thousand video clips collected from 10 countries.
      I think this is an import work and I might want to familiarize myself with
      the mentioned toolboxes and make use of the data set they provided.

      \item QuEst \cite{QuEst}: I still need to implement this in Python.

      \item VEst \cite{dani2009position}: I am working my way through this paper.
      I reviewed Homography and SVD in detail. I went over EKF as well.

      \item NASA MSI Fellowship: Need to read more NASA papers.

      \item PyTorch Tutorials: Transfer learning:


      \item NBV Grasping Project: On Wednesday, Joe helped me build UR driver
      code from source on my computer which is a great news. This allows us to
      proceed with ROS Noetic which has Python3 natively integrated. Moreover,
      I am working on updating URDF file to meet our needs. Today, I am updating
      rvl-workstation with Ubuntu 20.04 and ROS Noetic. I also created a channel
      on Teams where we keep a record of our discussions and decisions. Additionally,
      I created a MS OneNote that allows us to seemlessly share notes and development
      documentations. OneNote and GitHub for this project are accessable through
      NBV-Grasping channel on Teams.

      \item UTARI:

      \item Implement features from PoseCNN, DOPE, and BayesOD. - On pause.
\end{itemize}

%\newpage

\section{Immediate Plans - Summer 2021:}
The following items are listed in the order of priority:

\begin{itemize}
      \item Pose estimation:
      \item NBV-Grasping:
      \item UTARI:
\end{itemize}

\section{Intermediate Goals - Fall 2021:}
\begin{itemize}
      \item Pose estimation: I must be finished with implementation, perhaps
      make some improvements, and should be working on a paper for ICRA or CVPR.
      \item Scene understanding and active learning: After pose estimation, I
      want to expand my research into scene understanding and active learning in
      the context of advanced manufacturing.
      \item ARIAC: Once I am up to speed, I will do the ARIAC workshops/tutorials
      and will talk to Jerry about possible contributions.
\end{itemize}


%Sets the bibliography style to UNSRT and import the
\newpage
\bibliography{references}
\bibliographystyle{ieeetr}

\end{document}
