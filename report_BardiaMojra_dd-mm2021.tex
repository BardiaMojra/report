\documentclass[11pt]{article}
\usepackage{bookmark}
\usepackage{algorithm}
\usepackage{algpseudocode}
\usepackage{amsfonts}
\usepackage{amsmath}
\usepackage{amssymb}
\usepackage{amsthm}
\usepackage{bm}
\usepackage{color}
\usepackage{comment}
\usepackage{float}
\usepackage{graphicx}
%\usepackage[hidelinks]{hyperref}
\usepackage{makecell}
\usepackage[caption=false,font=footnotesize,subrefformat=parens,labelformat=parens]{subfig}
\usepackage{wrapfig}
\usepackage{url}
\usepackage[table]{xcolor}
%
\setlength{\parindent}{0.25in}
\setlength{\parskip}{.05in}
\pagestyle{plain}
%Title, date an author of the document
\title{Progress Report}
\author{Bardia Mojra}


\begin{document}
\maketitle
\thispagestyle{empty}

\bigskip
\bigskip
\begin{center}
      Robotic Vision Lab
\end{center}

\begin{center}
      The University of Texas at Arlington
\end{center}

\newpage

\section{To Do}
\begin{itemize}
      \item MSI Fellowship: Read NASA papers. Develop proposal package.
      \item Setup ROC Client on UR5e.
      \item PyTorch tutorials: On-going.
      \item Implement a dense pose estimation algorithm with keypoint estimation:
      next.
      \item UR5e: Finish ROS Industrial tutorials.
      \item Look into Berk Calli's work \cite{calli2015ycb}.
      \item PVNet implementation: Paused.
      \item Normalized objects \cite{Wang_2019_CVPR}.
      \item Universal pose estimation.
      \item Look into methods of generating uncertainty data.
      \item Vision-based robotic grasping from object localization, object pose
      estimation to grasp estimation for parallel grippers - a review,
      \cite{du2020vision}: Will read after PVNet implementation.
      \item Look into PyBullet for RL.
      \item Look into Facebook Flashlight C++ library, \cite{flashlig35:online}.
      \item Look into Nvidia Omniverse, \cite{NVIDIAOm1:online}.
\end{itemize}

\section{Reading List}
\begin{itemize}
      \item \cite{roadmap251:online}
      \item \cite{ferraz2014leveraging}
      \item \cite{he2015deep}
      \item \cite{du2020vision}
\end{itemize}

\section{Progress}
The following items are listed in the order of priority:
\begin{itemize}
      \item I did tutorials on PIL, Pickle
      \item NASA MSI Fellowship: Next, I will read papers from NASA
      \cite{NASATech44:online} and develop a proposal.
      \item PyTorch Tutorials: Next: Transfer learning.
      \item PVNet: Next: Use transfer learning and ResNet to train a model for
      semantic segmentation on YCB dataset.
      \item NBV Grasping Project: Next, I will install ROS client on UR5 and
      my lab station.
      \item UTARI: No new development.
      \item Implement features from PoseCNN, DOPE, and BayesOD. - On pause.
\end{itemize}

%\newpage

\section{Immediate Plans - Summer 2021:}
The following items are listed in the order of priority:

\begin{itemize}
      \item Algorithms and Data Structures: I need to strengthen my algorithms
      and data structure implementation skills. I have
      noticed that my implementation could be a lot cleaner and could perhand
      process data more efficiently. I will focus on commonly used algorithms
      as I believe proficiency on this topic will decrease my average
      development time and make my code less prone to software bugs.
      \item Pose estimation: First, I will implement a simple pose estimation
      model and gradually will add feature extraction and other techniques for
      robust and fast pose estimation. I will have to learn how to extract
      features from ground truth data, i.g. label, bounding box, center, position,
      orientation and more. Some of these features are given but some need to
      calculated. So far, I have become familiar with Python development
      environment and 2D data manipulation. Next, I need to seek guidance on
      how to process 3D data sets. Coupled with what I have learned in CSE-6363
      and PyTorch tutorials, I am confident I can quickly develop a simple
      pose estimation model and improve it over the summer. I want to start
      writing a paper on this topic but it is difficult to set a timeline
      without a working implementation. Right after finals, I will resume working
      on this and read paper from CVPR and ICRA on the topic.
      \item NBV-Grasping: I will follow up with Chris and Joe and will try to
      assist and learn as much as I can. The goal is to write the paper by mid
      to end of the summer.
      \item UTARI: It depends on Dr. Gans' plan for the summer. Most likely, I will
      be working on phased array radar project.
\end{itemize}

\section{Intermediate Goals - Fall 2021:}
\begin{itemize}
      \item Pose estimation: I must be finished with implementation, perhaps
      make some improvements, and should be working on a paper for ICRA or CVPR.
      \item Scene understanding and active learning: After pose estimation, I
      want to expand my research into scene understanding and active learning in
      the context of advanced manifacturing.
      \item ARIAC: Once I am up to speed, I will do the ARIAC workshops/tutorials
      and will talk to Jerry about possible contributions.
\end{itemize}


%Sets the bibliography style to UNSRT and import the
\newpage
\bibliography{references}
\bibliographystyle{ieeetr}

\end{document}
