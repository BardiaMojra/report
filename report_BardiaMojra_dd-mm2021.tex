\documentclass[11pt]{article}
\usepackage{bookmark}
\usepackage{algorithm}
\usepackage{algpseudocode}
\usepackage{amsfonts}
\usepackage{amsmath}
\usepackage{amssymb}
\usepackage{amsthm}
\usepackage{bm}
\usepackage{color}
\usepackage{comment}
\usepackage{float}
\usepackage{graphicx}
%\usepackage[hidelinks]{hyperref}
\usepackage{makecell}
\usepackage[caption=false,font=footnotesize,subrefformat=parens,labelformat=parens]{subfig}
\usepackage{wrapfig}
\usepackage{url}
\usepackage[table]{xcolor}
%
\setlength{\parindent}{0.25in}
\setlength{\parskip}{.05in}
\pagestyle{plain}
%Title, date an author of the document
\title{Progress Report}
\author{Bardia Mojra}


\begin{document}
\maketitle
\thispagestyle{empty}

\bigskip
\bigskip
\begin{center}
      Robotic Vision Lab
\end{center}

\begin{center}
      The University of Texas at Arlington
\end{center}

\newpage

\section{To Do}
\begin{itemize}
      \item Implement pose estimation: Implement PVNet, based on ResNet.
      \item Implement pose estimation: Train PE on YCB data.
      \item Implement DOPE with added dropout before each layer to estimate
            variational Bayesian inference (not sure if this is applicable anymore).
      \item Implement PoseCNN, DOPE, and BayesOD. (No longer applicable - maybe
      for pose estimation paper?)
      \item Pose Estimation Servery: Working on different methods, Hollistic,
      Dense, and Keypoint-based.
      \item Vision-based robotic grasping from object localization, object pose
      estimation to grasp estimation for parallel grippers - a review,
      \cite{du2020vision}: Will read after PVNet implementation.
\end{itemize}

\section{Reading List}
\begin{itemize}
      \item \cite{du2020vision}
      \item \cite{he2015deep}
      \item \cite{ferraz2014leveraging}
\end{itemize}


\section{Progress}
The following items are listed in the order of priority:
\begin{itemize}
      \item Pose Estimation: Thank you for this paper. I read and fully
      annotated the paper. For a single shot, this is exactly what I was thinking
      about. I have been trying to implement the clean-pvnet version from authors'
      repository with no avail so. Last night, I installed CUDA-9.0 to run the
      project but it failed when I tested the provided sample code. When I started
      my computer this morning, it asked for updates related to GCC-6 and NVidia
      which are both related to CUDA-9.0. I will try to get to work one more time,
      then I will move on to the provided Docker implementation. That should be
      fairly simple. Then, I will start implementing the code as a base, I already
      have some ideas for improvements. Their pixel-wise voting scheme has no
      analytical supporting evidence as to why it should work or why it works.
      Besides that, I think it is very well written paper. After reading it,
      I think I have a clear picture of what to write on for the pose estimation
      survey paper. I started based on PVNet paper, but instead of explanding the
      related work here, I am developing it under pose estimation survey paper.

      \item PVNet \cite{peng2019pvnet}:Although 6D pose estimation has been the
      subject of research for many
      years and great accuracy has been achieved, many states of the art solutions
      do not take advantage of uncertainty among observed features.
      In this paper, the authors propose a novel two-stage pose estimation
      framework, Pixel-wise Voting Network or PVNet, which generates and uses
      keypoint features uncertainty data. First, they estimate the 2D
      keypoints for each object in a RANSAC-like fashion which enables uncertainty
      measurement in the following stage. In the second stage, they use a
      modified EPnP algorithm \cite{lepetit2009epnp} that leverages feature
      uncertainty \cite{ferraz2014leveraging} to calculate estimated object 6D
      position.

      \item YCB Dataset \cite{calli2015ycb}: PVNet used a trained ResNet as a
      base for their model. I will follow their example as they used YCB dataset
      as well. More importantly, instead of Object Detections, they used ResNet's
      segmentation feature which allows for pixel-wise voting (and other
      uncertainty-type implementations).

      \item Normalized Objects \cite{Wang_2019_CVPR}:
      \item Implement features from PoseCNN, DOPE, and BayesOD.
\end{itemize}

%\newpage

\section{Plans}
The following items are listed in the order of priority:

\begin{itemize}
      \item Pose Estimation in Simulation \cite{NVIDIAIs75:online}: Use Nvidia
      Isaac SDK for in-simulation pose estimation training.
      \item Look into domain randomization and adaptation techniques.
      \item Project Alpe with Nolan: On pause for right now.
      \item UR5e: Finish ROS Industrial tutorials.
\end{itemize}

\section{2021 Goals and Target Journals/Conferences}
\begin{itemize}
      \item Submit a paper on pose estimation with uncertainty to ICIRS.
      \item Get comfortable with TensorFlow and related Python modules.
      \item Keep writing.
\end{itemize}


%Sets the bibliography style to UNSRT and import the
\newpage
\bibliography{references}
\bibliographystyle{ieeetr}

\end{document}
