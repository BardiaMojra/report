\documentclass[11pt]{article}
\usepackage{bookmark}
\usepackage{algorithm}
\usepackage{algpseudocode}
\usepackage{amsfonts}
\usepackage{amsmath}
\usepackage{amssymb}
\usepackage{amsthm}
\usepackage{bm}
\usepackage{color}
\usepackage{comment}
\usepackage{float}
\usepackage{graphicx}
%\usepackage[hidelinks]{hyperref}
\usepackage{makecell}
\usepackage[caption=false,font=footnotesize,subrefformat=parens,labelformat=parens]{subfig}
\usepackage{wrapfig}
\usepackage{url}
\usepackage[table]{xcolor}
%
\setlength{\parindent}{0.25in}
\setlength{\parskip}{.05in}
\pagestyle{plain}
%Title, date an author of the document
\title{Progress Report}
\author{Bardia Mojra}


\begin{document}
\maketitle
\thispagestyle{empty}

\bigskip
\bigskip
\begin{center}
      Robotic Vision Lab
\end{center}

\begin{center}
      The University of Texas at Arlington
\end{center}

\newpage

\section{Specific Research Goals}
\begin{itemize}
      \item Grant Proposal (Oct 14th): Keep working on this.
      \item VPQEKF (IROS - Mar. 1st): Work on the paper, focus on this in October.
      \item NBV-Grasping (IROS - Mar. 1st): Work on tasks assigned by Chris, one day a week. Focus on this from November till March.
\end{itemize}

\section{To Do}
\begin{itemize}
  \item Grant Proposal: Find three scientific questions to be researched. - Done
  \item Grant Proposal: Finish the first complete draft of the proposal over the weekend.
  \item Grant Proposal: Next week, start working on the personal statement, use UTA SOP as the initial draft.
  \item PVQEKF:
  \begin{itemize}
      \item Go over code and write matrix equations.
      \item I will go over the paper once every morning and expand sections for 30 minutes to an hour.
      \item Double-check my data prep implementation. Use KITTI Python module.
      \item Test with Hilti dataset.
      \item Add L2-norm and L2 loss features.
      \item I need to separate the state observation and control input vectors from the
      \item Develop object tracking and robust-to-truncation feature.
      \item Get ROS environment up and running. I need to install Armadillo (C++) with a certain dependency configuration.
  \end{itemize}

  \item Real-time pose estimation demo.
  \item NBV-Grasping:
      \begin{itemize}
      \item Update URDF and Xacro files for UR5e to include a sensor, sensor mount (with offset), and the gripper. -- Next
      \item Add movement constraints for tables and scenes.
      \item Write two IK functions for gripper and sensor, one for each. It should plug-in with MoveIt configurator.
      \item Research and implement point-cloud data to training TensorFlow models.
      \item Learn and implement GraspIt package.
      \end{itemize}
\end{itemize}


\section{Progress}
The following items are listed in the order of priority:
\begin{itemize}
      \item Fellowship: Here is a selection of what I have been working on.

      \textbf{Background:} Automation and advanced manufacturing in the automotive
      industry has been researched for decades and yet the wiring harness production
      and integration remains up to 90\% dependant on manual labor \cite{nguyen2021manufacturing}.
      With the emergence of electric vehicles (EV), the wiring harness has become
      a safety-critical component as it is responsible for power delivery, steering,
      braking, and sensing.

      To achieve higher levels of safety and quality,
      more fundamental scientific tools must be developed for adequate handling of wiring harness tasks.
      The wiring harness problem can be characterized by manipulation,
      bundling, cavity insertion or handling of deformable linear objects or Kirchhoff's elastic rods as it is known in the literature.

      The automotive wiring harness has a tree-like structure with thousands of wire pieces, components, terminals, etc \cite{Log-linearLearnMan}.
      Wires with different gauges and lengths are bundled together to form the trunk of the tree and branches deliver power and data connection to the submodules such as exterior cameras, entertainment systems, brakes, etc.

      In recent years, there has been a growing interest in achieving autonomous manufacturing and technologies with robotic manipulation be among the highly researched topics. Although robotic manipulators have found a prominent place on the production line, the automotive wire harnessing task remains up to 90\% dependant on manual labor \cite{nguyen2021manufacturing}.

      Due to low and fluctuating volumes, automotive manufacturers tend to implement some form of JIT (just-in-time) production to avoid logistical complications.

      \noindent
      \textbf{Bridge Background to Problem Statement}
      Degrees of automation deployed vary among the three stages with the first being most heavily automated and the last being the least automated. The human-machine production rate disparity is clear here\\
      Current elastic rod manipulation approaches could be categorized to two main groups, classical and data-driven. Classical methods such as...\\

      Currently, 70\% of wiring harness manufacturing time is spent on the final step and in some areas it is up to 90\% manual labor \cite{nguyen2021manufacturing}.

      \noindent
      \textbf{Hypothesis:} Although there has been considerable effort made towards resolving a multitude
      of issues regarding handling of deformable linear objects, there are gaps in
      our fundamental scientific toolbox. These include but not limited to, 1) the lack of
      necessary engineering tools for rapid reconfiguration and deployment of such
      systems. In 1996, Kavraki introduced a method, \cite{pRoad96}, for computing
      \emph{probabilistic roadmaps for path-planning in high-dimensional configuration space}
      for stationary workspaces. In her method, first a sparse map is computed offline
      and stored as a graph whose nodes corresponds to collusion-free configurations and
      whose edges represent collusion-free paths between the nodes. Then, while in
      operation the robot searches for a path connecting its initial to destination
      configuration.

      \noindent
      In \cite{briggsWire}, Briggs for modeling a Krichhoff's elastic rod by considering a dual-arm approach to form
      a closed-loop geometric control

      \noindent
      In \cite{Quasi-static}, Bretl and McCarthy expanded Briggs' approach by reducing
      high-dimensional space representation of Krichhoff's elastic rods to $\R^6$ [SOURCE??]
      Moreover, they showed that the set of equilibrium configurations for an elastic rod
      is a smooth manifold of finite dimension that can be represented by a single global chart.

      data set available to researchers, effective tools
      \noindent
      \textbf{Question 1 - Current Limitations - }

      Current state-of-the-art technologies are difficult to implement and computation-heavy to reconfigure to even deploy an existing mapping for cable handling in the production environment.

      \noindent
      \textbf{Question 2 - Knowledge Gaps - Empirical and Simulated Data Set}

      \noindent
      \textbf{Question 3 - Implementation Difficulties}

      Moreover, it is essential to develop easily reconfigurable wire harnessing manufacturing processes and systems. Currently, it involves long computational processes to model and tune or perform transfer learning techniques.


      \noindent
      Despite the lack of long research history in this area, recent development have shown promising results. [cite cable harnessing papers]\\

      Relative developments that makes my proposal ideal next step..\\

      Multiple model based and model-free systems have been develop for robotic are
      manipulation.
      Current limitations?
      How can our contribution advance SOTA? 
      Assumption is that the research is not done yet! We don’t know the solution and that's okay.

      mention the gap\\
      how are we training\\
      Cost functions and rewards\\

      \item VPQEKF: No update. I haven't done much with Hilti dataset after downloading it.
      \item NBV Grasping Project: No updates.
      \item PyTorch Tutorials: Transfer learning.
      \item Pose Estimation: On pause.
      \item SD Team: No update.
      \item EE Autonobots: No update.
\end{itemize}

%\newpage

\section{Intermediate Goals - Fall 2021:}
\begin{itemize}
      \item QEKF: Finish paper.
      \item Active Learning.
      \item ARIAC: Once I am up to speed, I will do the ARIAC workshops/tutorials and will talk to Jerry about possible contributions.
\end{itemize}


%Sets the bibliography style to UNSRT and import the
\newpage
\bibliography{ref}
\bibliographystyle{ieeetr}

\end{document}
