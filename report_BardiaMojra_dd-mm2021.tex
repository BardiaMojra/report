\documentclass[11pt]{article}
\usepackage{bookmark}
\usepackage{algorithm}
\usepackage{algpseudocode}
\usepackage{amsfonts}
\usepackage{amsmath}
\usepackage{amssymb}
\usepackage{amsthm}
\usepackage{bm}
\usepackage{color}
\usepackage{comment}
\usepackage{float}
\usepackage{graphicx}
%\usepackage[hidelinks]{hyperref}
\usepackage{makecell}
\usepackage[caption=false,font=footnotesize,subrefformat=parens,labelformat=parens]{subfig}
\usepackage{wrapfig}
\usepackage{url}
\usepackage[table]{xcolor}
%
\setlength{\parindent}{0.25in}
\setlength{\parskip}{.05in}
\pagestyle{plain}
%Title, date an author of the document
\title{Progress Report}
\author{Bardia Mojra}


\begin{document}
\maketitle
\thispagestyle{empty}

\bigskip
\bigskip
\begin{center}
      Robotic Vision Lab
\end{center}

\begin{center}
      The University of Texas at Arlington
\end{center}

\newpage

\section{To Do}
\begin{itemize}
      \item PVNet implementation: Test and document, learn and rewrite.
      \item Implement pose estimation: Keypoint uncertainty, understand RANSAC.
      \item Look into methods of generating uncertainty data.
      \item Pose Estimation Servery: On pause.
      \item Vision-based robotic grasping from object localization, object pose
      estimation to grasp estimation for parallel grippers - a review,
      \cite{du2020vision}: Will read after PVNet implementation.
\end{itemize}

\section{Reading List}
\begin{itemize}
      \item \cite{ferraz2014leveraging}
      \item \cite{he2015deep}
      \item \cite{du2020vision}
\end{itemize}

\section{Progress}
The following items are listed in the order of priority:
\begin{itemize}
      \item Pose Estimation, PVNet \cite{peng2019pvnet}: Today, I sat down with
      Joe, we ran PVNet Docker and tried to run the tests and recreated the
      same issue. It turns out Cuda 9 is not supported on my GPU (RTX-2060) and
      he suggested I run it on lab computer. I looked into that, it seems to
      have weird issue with Cuda again where it thinks it is installed but it
      cannot find it. Quan suggested I implement the code from scratch and I
      agree. After recent assignments from Dr. Huber's class I feel much more
      confident implementing from scratch.

      PVNet architecture is based on a pretrained ResNet-18 \cite{he2015deep}
      and consists of 9 sections, where each section consists of two to three
      layers. First section consists of a Conv-BN-ReLU followed by a max pooling
      layer, with skip connection to ninth section. Second section consists of
      two residual blocks with skip connection to the eighth section. Third
      section consists of a residual block with stridden convolutions with
      skip connection to the seventh block. Fourth section consists of a
      residual block with stridden convolutions followed by a residual block
      with a skip connection to the sixth section. Fifth section consists of two
      residual blocks with dilated convolutions. Sixth section consists of
      two conv-BN-ReLU layers followed by a bilinear upscaling layer. Seventh
      and eighth sections each consist of a conv-BN-ReLU layer followed by a
      bilinear upsampling layer. This network outputs both semantic labels and
      vector-field predictions estimating center of the object.


      I read on RANSAC \cite{fischler1981random}, it is very straight forward.
      There are many variations of it and I found Neural-Guided RANSAC
      \cite{brachmann2019neural} which allows for optimization of arbitrary task
      loss functions. They claim it results in large improvement on classic
      visual tasks.

      PyTorch: I started doing some tutorials on PyTorch.

      \item YCB Dataset \cite{calli2015ycb}: Start with YCB data and look into
      Berk Calli's work.
      \item Normalized Objects \cite{Wang_2019_CVPR}:
      \item Implement features from PoseCNN, DOPE, and BayesOD. - On pause.
\end{itemize}

%\newpage

\section{Plans}
The following items are listed in the order of priority:

\begin{itemize}
      \item Pose Estimation in Simulation \cite{NVIDIAIs75:online}: Use Nvidia
      Isaac SDK for in-simulation pose estimation training.
      \item Look into domain randomization and adaptation techniques.
      \item Project Alpe with Nolan: On pause for right now.
      \item UR5e: Finish ROS Industrial tutorials.
\end{itemize}

\section{2021 Goals and Target Journals/Conferences}
\begin{itemize}
      \item Submit a paper on pose estimation with uncertainty to ICIRS.
      \item Get comfortable with TensorFlow and related Python modules.
      \item Keep writing.
\end{itemize}


%Sets the bibliography style to UNSRT and import the
\newpage
\bibliography{references}
\bibliographystyle{ieeetr}

\end{document}
