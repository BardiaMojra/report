\documentclass[11pt]{article}
\usepackage{bookmark}
\usepackage{algorithm}
\usepackage{algpseudocode}
\usepackage{amsfonts}
\usepackage{amsmath}
\usepackage{amssymb}
\usepackage{amsthm}
\usepackage{bm}
\usepackage{color}
\usepackage{comment}
\usepackage{float}
\usepackage{graphicx}
%\usepackage[hidelinks]{hyperref}
\usepackage{makecell}
\usepackage[caption=false,font=footnotesize,subrefformat=parens,labelformat=parens]{subfig}
\usepackage{wrapfig}
\usepackage{url}
\usepackage[table]{xcolor}
%
\setlength{\parindent}{0.25in}
\setlength{\parskip}{.05in}
\pagestyle{plain}
%Title, date an author of the document
\title{Progress Report}
\author{Bardia Mojra}


\begin{document}
\maketitle
\thispagestyle{empty}

\bigskip
\bigskip
\begin{center}
      Robotic Vision Lab
\end{center}

\begin{center}
      The University of Texas at Arlington
\end{center}

\newpage

\section{Specific Research Goals}
\begin{itemize}
      \item Grant Proposal: Controls-Deep Learning Hybrid Systems for Industrial Cable Harnessing Applications. Focus on this for September (hard deadline).
      \item VPQEKF (IROS - Mar. 1st): Work on the paper, focus on this in October.
      \item NBV-Grasping (IROS - Mar. 1st): Work on tasks assigned by Chris, one day a week. Focus on this from November till March.
\end{itemize}

\section{To Do}
\begin{itemize}
  \item Grant Proposal: I will work on this every day for the next 2-3 weeks, then I will move back on VPQEKF project.
  \item PVQEKF:
  \begin{itemize}
      \item Write equations in LateX with description. --- Done.
      \item I will go over the paper once every morning and expand sections for 30 minutes to an hour.
      \item Double-check my data prep implementation. Use KITTI Python module.
      \item Test with Hilti dataset.
      \item Add L2-norm and L2 loss features.
      \item I need to separate the state observation and control input vectors from the
      \item Develop object tracking and robust-to-truncation feature.
      \item Get ROS environment up and running. I need to install Armadillo (C++) with a certain dependency configuration.
  \end{itemize}

  \item Real-time pose estimation demo.
  \item NBV-Grasping:
      \begin{itemize}
      \item Update URDF and Xacro files for UR5e to include a sensor, sensor mount (with offset), and the gripper. -- Next
      \item Add movement constraints for tables and scenes.
      \item Write two IK functions for gripper and sensor, one for each. It should plug-in with MoveIt configurator.
      \item Research and implement point-cloud data to training TensorFlow models.
      \item Learn and implement GraspIt package.
      \end{itemize}
\end{itemize}

\section{Reading List}
\begin{itemize}
      \item Leveraging feature uncertainty in the pnp problem \cite{ferraz2014leveraging}.
      \item Normalized objects \cite{Wang_2019_CVPR}.
      \item NASA papers \cite{NASATech44:online}.
\end{itemize}



\section{Progress}
The following items are listed in the order of priority:
\begin{itemize}
      \item Fellowship: I have already prepared an execution plan with steps and instructions. I prioritize this task moving forward until the end of the month. I am working on dissecting \cite{sintov2020motion}, \cite{doi:10.1177/0278364912473169}, and \cite{bergou2008discrete}. \cite{doi:10.1177/0278364912473169} and \cite{bergou2008discrete}provide a relatively simple framework for modeling and estimating system behavior for \textit{discrete elastic rods}. \cite{sintov2020motion} introduces a straightforward and effective approach for manipulating cables and wires for harnessing applications. It is my understanding that the source code for these papers is available.
      \item VPQEKF: I continued working on the paper. It is still in the early stages; I will continue to review it and provide updates regularly. I wrote a section on Quaternion Algebra based on a book with the same name. It is added to this report in the next section. I also started working on the Hilti dataset.
      \item NBV Grasping Project: No updates. We are going to work on this project every Friday afternoon.
      \item PyTorch Tutorials: Transfer learning.
      \item Pose Estimation: On pause.
      \item SD Team: No update.
      \item EE Autonobots: No update.
\end{itemize}

\section{QUATERNION ALGEBRA} \label{sec:QuatAlg}

Quaternion space is a non-minimal representation belonging to \textit{SO(3)} Lie group.

\subsection{Unit Quaternion}
Moreover, the quaternion term from the dataset has \textit{four terms} with $xyzw$
format.
Hamilton's quaternion defined by 3 perpendicular imaginary axes $i,j,k$ with
real scalars $x,y,z$ and a real term $w$ which constraints other 3 dimension to
a \textit{unit magnitude}. Thus, the fourth term normalizes the vector's magnitude
conveniently and preserves the 3D rotation (3 DOF). We define \textbf{Unit Hamiltonian}
or \textbf{Unit Quaternion} as,


\begin{eqnarray}\nonumber
\label{eq:9}
\mathbb{H}^{1} &:=&\left\{ q_{wxyz}=w+xi+yj+zk~\in \mathbb{H}~| \right.\\
                   && \left.~w^{2}+x^{2}+y^{2}+z^{2}=1 \right\}
\end{eqnarray}
% page 27, sec 2.4.2 of "Quaternion Algebra"

Where superscript 1 in $\mathbb{H}^{1}$ denotes a unit quaternion space with 4
terms. There are two equal representations for $\mathbb{H}^{1}$ subgroup; thus,
we provide a concise definition and notation for both to avoid confusion. The first representation is shown in \ref{eq:5} where the four terms of the
quaternion are arranged in $wxyz$ order and it is represented by $q_{wxyz}$.
The second quaternion is arranged in $xyzw$ format and is represented by $q_{xyzw}$.
It is important to note the difference as both are used in our derivation and
implementation.

\begin{equation}
\label{eq:12}
q_{wxyz} = q_{xyzw} ~; ~~~ q_{wxyz},~q_{xyzw} \in \mathbb{H}^{1}
\end{equation}


\subsection{Pure Quaternion}
As previously mentioned, the three imaginary terms of the quaternion represent the
3D angles of interest in radians and the fourth dimension constraints the vector
magnitude. Thus to avoid computational errors, in the prediction step, we use
the unit quaternion where it only has its three imaginary terms, $xyz$.
This quaternion space representation is defined by $\mathbb{H}^{0}$ and
denoted by $q_xyz$ variables.



\begin{eqnarray}\nonumber
\label{eq:11}
\mathbb{H}^{0} &:=& \left\{ q_{xyz}=xi+yj+zk~ \in \mathbb{H}~| \right.\\
&& \left. ~x,y,z \in \mathbb{R} \right\} \backsimeq  \mathbb{R}^3
\end{eqnarray}
% page 27, sec 2.4.2 of "Quaternion Algebra"



\subsection{Exponential Map}

For calculating incremental rotation in

Incremental rotation estimation using the skew-symmetric matrix obtained form
the rotational rate vector and matrix exponential mapping function, [QEKF01].
Gamma, $\Gamma$, represents incremental

\begin{equation}
\label{eq:2}
\Gamma_{0} := \sum_{i=0}^{\infty} \frac{ \left( \Delta t^{i+n}  \right)}{ \\
\left( i + n \right) \! } \\
\omega^{\times i},
\end{equation}

Where $(.)^{\times}$ represents skew-symmetry matrix of a vector


\subsection{Updating Quaternion State}
\begin{equation}
\label{eq:13}
q_{i+1} = \delta q_{i} \otimes \widehat{q}_{i}
\end{equation}



\subsection{Capturing Quaternion Error}
We use the mapping function $\zeta(.)$ to calculate the quaternion
state error from the error rotation vector, [QEKF01].

\begin{equation}
\label{eq:14}
\delta q = \zeta(\delta \phi),
\end{equation}

\begin{equation}
\label{eq:15}
\zeta : v \rightarrow \zeta(v) =
        \begin{bmatrix}
        \sin(\frac{1}{2}\|v\|) \frac{v}{\|v\|} \\
        \cos(\frac{1}{2}\|v\|)
        \end{bmatrix}
\end{equation}



%\newpage

\section{Intermediate Goals - Fall 2021:}
\begin{itemize}
      \item QEKF: Finish paper.
      \item Active Learning.
      \item ARIAC: Once I am up to speed, I will do the ARIAC workshops/tutorials and will talk to Jerry about possible contributions.
\end{itemize}


%Sets the bibliography style to UNSRT and import the
\newpage
\bibliography{ref}
\bibliographystyle{ieeetr}

\end{document}
