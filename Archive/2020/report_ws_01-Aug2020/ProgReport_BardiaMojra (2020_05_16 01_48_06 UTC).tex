\documentclass[11pt]{article}
\usepackage{algorithm}
\usepackage{algpseudocode}
\usepackage{amsfonts}
\usepackage{amsmath}
\usepackage{amssymb}
\usepackage{amsthm}
\usepackage{bm}
\usepackage{color}
\usepackage{comment}
\usepackage{float}
\usepackage{graphicx}
\usepackage[hidelinks]{hyperref}
\usepackage{makecell}
\usepackage[caption=false,font=footnotesize,subrefformat=parens,labelformat=parens]{subfig}
\usepackage{wrapfig}
\usepackage{url}
\usepackage[table]{xcolor}


\setlength{\parindent}{0.25in}
\setlength{\parskip}{.05in}
\pagestyle{plain}

%Title, date an author of the document
\title{Progress Report}
\author{Bardia Mojra}


\begin{document}
	\maketitle
	\thispagestyle{empty}

\begin{center}
\bigskip
\bigskip
 Robotic Vision Lab\\
 The University of Texas at Arlington\\
\end{center}

\newpage 

\section{Progress}
Following items are listed in order of priority: 
\begin{itemize}  
	
  \item I have gone through ROS tutorials till end of chapter 1.1.5, understanding ROS nodes. I expect to finish chapter 1.1, beginner level, by next week. I will finish chapter 1.2 by the following week. 
  
  \item I have begun reading Robotics Vision and Control book from part 2 which covers mobile robots and SLAM. The code and examples are written for MATLAB and Peter Corke provides good MATLAB resources but I will try to implement the examples in python using Numpy and other module. I believe this would be more aligned with my goals, might be wrong. I still need to find a good source on sensor fusion. 
  
  \item Continue on TurtleBot SLAM tutorials. 
  
  \item My AI and Robotic goals for the summer comprise of learning ROS, Gazebo, and SLAM and begin working towards a model that can compete in Niko Suenderhauf's Scene Understanding Challenge. Other than that, I will work on finishing everything on this list. 
    
  \item I still need to work on fellowship applications, I will write a new draft by next week. 
    
  \item Still working on \cite{PanopticSeg2019}, \cite{SVO}, \cite{HornsMethod}, \cite{NYUV2}, \cite{DGCNNLPC}, and \cite{MaskRCNN}.

\end{itemize}

\section{Plans}
Following items are listed in order of priority: 

\begin{itemize} 

  \item (Supremely important) Read on scene understanding, semantic SLAM, graph SLAM, visual odometer, place recognition, and Kalman Filtering. Read Niko Sunderhauf's research publications. 
  
  \item I will write a new draft by next week.	
 
 \item I will begin reading three paper a week. 
 
 \item Begin working on quaternions tutorial.  

  \item Get intimate with Python, Numpy, Pandas, Scipy, and Matplotlib, TensorFlow and PyTorch. 
  
  \item (On pause) Resume Machine Learning course with Andrew Ng as soon as possible.  
 
 \item (On pause) I am working through Jason Brownlee's ML Mastery book, \cite{MLmasteryPy}.
 
 \item (On pause) Resume Robotic Perception course as soon as possible.

 \item (On pause) Need to read \cite{ImSRwDeepCNN}, \cite{MixDNNforSISR}, \cite{mModalSemanticSLAMwProb}, and \cite{RCANforImClass}; these papers seem fundamental to understanding the overall picture.

 \item (On pause) Read Digital Image Processing by Gonzalez and Woods. 
 
 
 \end{itemize}


\section{General Notes}
This section summarizes general research leads. The following item are to be investigated, understood and briefly summarized. 

\begin{itemize}
	\item Open3D: An open source toolbox used for truth occupancy grid application and probably other things. Should review. 
	
	\item Horn's paper \cite{HornsMethod}: It introduces Unit Quaternions which allow for complex domain representation of kinematics. Very important paper for robotic motion. Should review.
	
	\item Bayesian Learning: This probabilistic ML approach treats model parameters as random variables. Read \cite{Bayesian_Learning_2012} for more details. 
	
	\item Convex Optimization:
	
	\item Q-Learning: A learning model of reinforcement (RL), learning from delayed reward. 
	
	\item Deep Reinforcement Learning: 
\end{itemize}

\section{Literature Review}

\subsection{Dynamic Graph CNN for Learning on Point Clouds \cite{DGCNNLPC}}
 This paper introduces a new model for training CNNs to learn similar features of point cloud objects. 

\subsubsection{Keywords}
\begin{itemize}
	\item PointNet
	\item Extrinsic and intrinsic descriptors:
	\item Permutation variance:
	
\end{itemize}

\subsection{Single Image Super-Resolution Using Multi-Scale Convolutional Neural Network - MSSR \cite{MSSRwCNN}}

  Paper proposes an architecture with two parallel path with different depths (which correspond to scales) for residual learning; where one path (module L) is used for large factor up-scaling (x4, x8) and the other (module S) for small factor up-scaling (x2). At the end, it combines the outputs by summation (a form of ensembles). In contrast to previous work where the focus is on small factor up-scaling (x2) and repeat if needed, this model takes higher factor up-scaling into consideration while training the network which helps with reducing blurriness of output image for higher factor up-scaling. The model uses multi-scale residual learning to train on general model for multiple up-scaling factors; hence, saving memory and processing time. This paper provides experiment results that show higher output image integrity where peak signal to noise ratio (PSNR) and structural similarity index (SSIM) are higher or comparatively close to state-of-the-art methods.  

\subsubsection{Keywords}
\begin{itemize}
	
	\item Lanczos re-sampling:
	
	\item Statistical priors:
	
	\item Stochastic Neighbor Embedding (paper by G. Hinton \cite{SNE}): Read paper.
	
	\item Parse coding:
	
	\item GoogLeNet:
	
	\item YCbCr color space: 
	
	\item Caffe package (paper by Jia, Y., Shelhamer, E., Donahue, J., Karayev, S., Long, J., Girshick, R., Guadarrama, S., Darrell, T.): Convolutional architecture for fast feature embedding. \underline{Read and write literature review}.  
	
	\item Adam method (paper by Kingma, D., Ba, J.): A method for stochastic optimization. 
	
	\item PSNR (performance metric): Peak Signal to Noise Ratio represents the ratio between max (peak) possible value (power) of a signal (image) and the power of distorting noise.  
	
	\item SSIM (performance metric): Structural Similarity Index, is the ratio of structural features of a processed image to the original image. The value represent percentage of structural features/information retained throughout image processing.  
	
	\item A+ (SR method, paper):
	
	\item SelfEx (SR method, paper):
	
	\item SRCNN (SR method, paper): Image Super Resolution Using Deep Convolutional Networks. \underline{Read again and write literature review}.  
	
	\item FSRCNN (SR method, paper):
	
	\item VDSR (SR method, paper):
	
\end{itemize} 


\subsection{Value Iteration Networks - VIN \cite{VIN}}
\subsubsection{Keywords}
\begin{itemize}
	\item Imitation Learning: **need to finish reading \cite{Vitaly_Kurin_2017}.**
	
	\item CNNs applied to reinforcement learning: 
	
	\item MDP (Markov Decision Process):
	
	\item VI Algorithm (Value Iteration): 
	
	\item SGD (stochastic gradient decent):
	
	\item Theano Code: "Theano is a Python library that allows you to define, optimize, and evaluate mathematical expressions involving multi-dimensional arrays efficiently. It can use GPUs and perform efficient symbolic differentiation." 
	
	\item Grid-World Domain:

\end{itemize}

%Sets the bibliography style to UNSRT and imports the 
\newpage 
\bibliography{bibfile_bm} 
\bibliographystyle{ieeetr}

\end{document}
