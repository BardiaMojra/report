\documentclass[11pt]{article}
\usepackage{bookmark}
\usepackage{algorithm}
\usepackage{algpseudocode}
\usepackage{amsfonts}
\usepackage{amsmath}
\usepackage{amssymb}
\usepackage{amsthm}
\usepackage{bm}
\usepackage{color}
\usepackage{comment}
\usepackage{float}
\usepackage{graphicx}
%\usepackage[hidelinks]{hyperref}
\usepackage{makecell}
\usepackage[caption=false,font=footnotesize,subrefformat=parens,labelformat=parens]{subfig}
\usepackage{wrapfig}
\usepackage{url}
\usepackage[table]{xcolor}
\graphicspath{{images/}}
\setlength{\parindent}{0.25in}
\setlength{\parskip}{.05in}
\pagestyle{plain}
%Title, date an author of the document
\title{Progress Report}
\author{Bardia Mojra}


\begin{document}
\maketitle
\thispagestyle{empty}

\bigskip
\bigskip
\begin{center}
 Robotic Vision Lab
\end{center}

\begin{center}
The University of Texas at Arlington
\end{center}

\newpage



\section{Progress}
The following items are listed in the order of priority:
\begin{itemize}
  \item DLO Dataset (\textcolor{red}{RAL}): Last week, I wrote a script to
  parse collected data into PNG images, point cloud frames, and binary data.
  I added some backgrounds to the data
  collection rig, set up the panda arm for manipulation, and worked on the paper.
  This week, I will finish the MoveIt tutorial on Motion Planning with ROS, record
  motions and record manipulation test episodes. The latest draft of the paper
  is attached for your attention.
  \item DLO Manipulation (\textcolor{red}{IROS}): \cite{abraham2017model}.\\
  \item XEst (\textcolor{red}{RAL ---}): No update.\\
  \end{itemize}

\section{Research Plan}
This section outlines my current research plan for the next 3 months, 6 months,
and 1 year. Moreover, I have included open projects and ideas to keep track of
them.\\
Target conferences: ICRA, IROS (March), CASE (Late Feb.), NIPS.\\
Target Journals: RAL, CVPR, CORAL.\\

\subsection{Research Plan:}
\begin{itemize}
  \item \textbf{3 months:} The primary objective will be to publish the DLO dataset paper,
  (\textcolor{red}{DLO-1}),
  finished my classes, and to meet my next Ph.D. milestone, the comprehensive exam.
  My goal is to submit the DLO dataset paper to IROS by March 1st.\
  \item \textbf{6 months:} Next, I want to explore using DMD as a method to retrieve the
  correct Quaternion solution for the QuEst method, (\textcolor{red}{QuEst-01}).
  I believe this testing this
  is fairly fast and I should be able to publish that paper fairly quickly. I
  believe the RAL would be an appropriate journal to target; we can discuss
  this further with Dr. Gans to get his input.\
  \item \textbf{1 year:} Next, I want to focus on (\textcolor{red}{PIKO-01}) as a method
  for fast online system identification. My aim is to confirm this method by
  comparing it against existing Koopman-based methods. In the following work, I
  will extend this method to control DLOs in real time (\textcolor{red}{DLO-02}).\
\end{itemize}

\subsection{Research Pipeline:}
\begin{itemize}
  \item DLO-01 (\textcolor{red}{IROS - March 1st, 2023}):
  DLO manipulation dataset with DLO configuration and gripper pose, as well
  as the gripper control input. Ideally, UR5 back-EMF current and bus voltage
  should be recorded. A DLO mount is introduced. A method for configuration
  estimation is introduced. Perhaps, a method for learning DLO dynamic can be
  trained and introduced.\
  \item QuEst-01 (\textcolor{red}{IROS}):
  Optimal transform solution for QuEst based on dominant mode decomposition (DMD).
  \item PIKO-01 (\textcolor{red}{TBD}):
  This work leverages DMD and Physics-Informed machine learning to extract
  low-dimensional coherent modal structures from dynamic data. This method
  will extend DMD-based approaches to include mixed basis functions. Moreover,
  this method will automatically try to find the best fit at a specified range
  of ranks. This method will be validated by comparing it against the existing
  Koopman-based MPC control schemes for VTOL-DIP method and introducing a method
  for controlling VTOL-TIP in simulation.
  This method will become the backbone of my Koopman-based MPC
  control research effort.\
  \item DLO-02 (\textcolor{red}{TBD}):
  This method will extend PIKO-01 to a control method for the DLO-01 dataset.

\end{itemize}
  \newpage

%Sets the bibliography style to UNSRT and import the
\newpage
\bibliography{ref}
\bibliographystyle{ieeetr}

\end{document}
