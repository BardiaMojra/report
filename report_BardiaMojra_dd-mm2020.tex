\documentclass[11pt]{article}
\usepackage{bookmark}
\usepackage{algorithm}
\usepackage{algpseudocode}
\usepackage{amsfonts}
\usepackage{amsmath}
\usepackage{amssymb}
\usepackage{amsthm}
\usepackage{bm}
\usepackage{color}
\usepackage{comment}
\usepackage{float}
\usepackage{graphicx}
%\usepackage[hidelinks]{hyperref}
\usepackage{makecell}
\usepackage[caption=false,font=footnotesize,subrefformat=parens,labelformat=parens]{subfig}
\usepackage{wrapfig}
\usepackage{url}
\usepackage[table]{xcolor}
%
\setlength{\parindent}{0.25in}
\setlength{\parskip}{.05in}
\pagestyle{plain}
%Title, date an author of the document
\title{Progress Report}
\author{Bardia Mojra}


\begin{document}
\maketitle
\thispagestyle{empty}

\bigskip
\bigskip
\begin{center}
Robotic Vision Lab
\end{center}

\begin{center}
The University of Texas at Arlington
\end{center}

\newpage
\section{Long-Term Goals}
\begin{itemize}
    \item Agile Manufacturing 
\end{itemize}

\section{Mid-Term Goals}
\begin{itemize}
    \item Bayesian Pose Estimation 
    \item ARIAC
    \item Pose Estimation Survey 
\end{itemize}

\section{To Do}
\begin{itemize}
    \item Dissect Kendall's \cite{kendall2017uncertainties}: in progress. 
    \item Write a literature review for \cite{kendall2017uncertainties}: Done.
    \item Dissect BayesOD \cite{harakeh2020bayesod}: in progress.
    \item Write a literature review for \cite{harakeh2020bayesod}: Done.
    \item Implement PoseCNN, DOPE, and BayesOD.
	\item Look into transfer learning. 
	\item Read and write a review for \cite{tobin2017domain}: Done. 
	\item Look into domain randomization and adaptation techniques.
	\item MoreFusion \cite{MoreFusion}: Wrote a literature review. Done.
	\item Reading list: \cite{lampinen2001bayesian} and \cite{li2019survey}.
    \item Read \cite{choi2013rgb}.
    \item Search for recent pose estimation survey papers.
    \item Generate new data set for PoseCNN and DOPE using UE4. 
	\item Read more papers on deep pose estimation. \item Learn to use UE4. 

\end{itemize}


\section{Progress}
Following items are listed in order of priority:
\begin{itemize}
	\item Pose Estimation with BNN's: I am looking into Yarin Gal's work, he is 
	a leading figure in Bayesian deep learning. His PhD thesis is on Uncertainty
	 in Deep Learning, \cite{gal2016uncertainty}, and was advised by Zoubin 
	Ghahramani. I am thinking if would be a good idea to go over his thesis.  

	\item Reading material: This week I read \cite{sunderhauf2018limits} and 
	\cite{kendall2017uncertainties}. I highlighted important sections of 
	\cite{sunderhauf2018limits} and I taking notes 

	Moreover, I 
	
	I am currently working on dissecting 
	\cite{kendall2017uncertainties} and write a review for it simultaneously. 
	Next, I will expand the review I have written for BayesOD and add more 
	details as I look into the source code. My goal is to do some research every
	morning and to end that with a page of written material. I will focus on 
	Bayesian vision as it offers probabilistic predictions by using per pixel
	Bayesian inference rather than deterministic output predictions based on MLE
	 parameters as currently used by CNN's. 
	
	
	\item Bayesian Pose Estimation: Current pose estimation models with 
	performance comparable with state of the art such as \cite{posecnn}, 
	\cite{Dope}, \cite{MoreFusion}, and \cite{DenseFusion} all use CNN's. 
	
	Although the use of CNN's remains a powerful tool as a mean to compute rich 
	feature maps by taking advantage of color channels; it has its short 
	comings, mainly that predictions are deterministic even though the model 
	is probabilistic. 
	
	Classical neural networks and CNN's use maximum likelihood to calculate 
	network weights and biases which derive network outputs. Such models are 
	represented by conditional PDF \( p(\textbf{y} | \textbf{x}, \theta) \) 
	trained on data \( D = \{ \textbf{x}_{i},\textbf{y}_{i} \}_{i = 1}^{N} \) 
	and converging to a set of learned parameters 
	\( \theta = \{ \textbf{W}, \textbf{b} \}\) where

	$$ p(D| \theta) = \prod_{i = 1}^{N}  p(\textbf{y}_{i} |\textbf{x}_{i}, \theta)  $$
	
	defines that PDF. For training these models, parameters \textbf{W} and 
	\textbf{b} are learned through back-propagation using Maximum Likelihood 
	Estimate (MLE) method, as defined by  
	
	$$ \hat{\theta}_{MLE} = \underset{\theta}{argMax} \sum_{i = 1}^{N} ln ~p(\textbf{y}_{i} | \textbf{x}_{i}, \theta) $$
	
	

	but in bayesian ---- 
	
	We start with the two basic rules for Bayesian machine learning, sum rule 
	and product rule.
	
	Sum rule: 
	
	$$ p(\textbf{X}) = \sum_{\textbf{y}_{i}}~ p(\textbf{y}_{i} ,\textbf{x}_{i})  $$  
	
	Product rule: 
	
	$$  p(\textbf{X}_{i}, \textbf{Y}_{i}) = p(x)p(\textbf{y}_{i}|\textbf{x}_{i}) $$ 
	
	
	Bayesian Learning: Learning is done by applying Bayes' rule to the current 
	state of knowledge as new evidence becomes available throughout the training
	 process. 

	$$ p(\theta | D, M) = \frac{p(D|\theta,M)~p(\theta|M)}{p(D|M)}  $$
	
	Where \( p(D| \theta, M) \) is the likelihood of parameters \( \theta \) in 
	model M. Prior belief or probability of parameters for given model M. And 
	posterior belief of \( \theta \) given model M and training data D is 
	represented by \( p(\theta | D, M) \).
	
	
	
	Bayesian Prediction: 
	
	$$ p(x| D, M) = \int p(x| \theta, D, M) p(\theta| D, M) d\theta $$
	
	Bayesian Model Comparison: 
	
	$$ p(M|D) = \frac{p(D|M)p(M)}{p(D)}$$ 
	
	On the other hand, Bayesian Neural Networks or BNN's infer posterior 
	probability distribution frame by frame by marginalizing new likelihood 
	over the distribution of parameters, hence converging to steady-state 
	posterior distribution. Moreover, Bayesian deep learning tools presented 
	in \cite{kendall2017uncertainties} and \cite{gal2016uncertainty} allows for 
	study and analysis of \textit{aleatoric} and \textit{epistemic} uncertainties 
	together in one frame work. 

	\textit{Aleatoric}:
	
	\textit{Epistemic} uncertainty is what a model does not know due to the 
	incompleteness of training data. Epistemic uncertainty decreases (but it 
	will never be equal to zero) as training data is expanded. It is also 
	referred to as \textbf{model uncertainty}.
	
	Learned attenuation: 
	
	
	
	\item ARIAC: For now, I will focus on implementing pose estimation and BayesOD implementations.
	
	\item Pose Estimation Survey Paper Feedback: On hold, I am working on
	Bayesian Pose Estimation.

Joint Probability: Joint probability \(p(D,W)\) relates prior belief to 
posterior belief given new evidence 

\( p(D)p(W|D) = p(D,W) [joint~prob.] = p(W)p(D|W) [conditional~prob.] \)

Posterior probability for a particular weight values, W, for given training data D. 

Bayesian approach to deep neural network learning allows for quantafing 
uncertainties related to model parameters as well as uncertainties rooted in 
model structure. Uncertainty 


	\item TensorFlow \cite{CVTF2}: I am still working through chapter 2. 


	\item Project Alpe with Nolan: On pause for right now. 
	\item UR5e: 
	\item Fellowship:
	\item System Identification Presentation:



\end{itemize}

%\newpage

\section{Plans}
Following items are listed in order of priority:

\begin{itemize}

	\item (On pause) Continue with ROS Industrial tutorials and documentation.

	\item (On pause) Resume Robotic Perception course as soon as possible.

	\item (On pause) Read Digital Image Processing by Gonzalez and Woods.

\end{itemize}



%Sets the bibliography style to UNSRT and import the
\newpage
\bibliography{bib_BardiaMojra_dd-mm2020}
\bibliographystyle{ieeetr}

\end{document}

