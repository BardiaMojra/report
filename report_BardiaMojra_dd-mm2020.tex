\documentclass[11pt]{article}
\usepackage{bookmark}
\usepackage{algorithm}
\usepackage{algpseudocode}
\usepackage{amsfonts}
\usepackage{amsmath}
\usepackage{amssymb}
\usepackage{amsthm}
\usepackage{bm}
\usepackage{color}
\usepackage{comment}
\usepackage{float}
\usepackage{graphicx}
%\usepackage[hidelinks]{hyperref}
\usepackage{makecell}
\usepackage[caption=false,font=footnotesize,subrefformat=parens,labelformat=parens]{subfig}
\usepackage{wrapfig}
\usepackage{url}
\usepackage[table]{xcolor}
%
\setlength{\parindent}{0.25in}
\setlength{\parskip}{.05in}
\pagestyle{plain}
%Title, date an author of the document
\title{Progress Report}
\author{Bardia Mojra}


\begin{document}
\maketitle
\thispagestyle{empty}

\begin{center}
	\bigskip
	\bigskip
	Robotic Vision Lab

	University of Texas at Arlington
\end{center}

\newpage

\section{Progress}
Following items are listed in order of priority:
\begin{itemize}

	\item Pose Estimation: This week I mostly worked on LTI system identification problem. We were given input and output signals of a second order LTI system and were tasked with modeling system parameters. Proposed solution uses Recursive Least Square Error algorithm which is implemented in ARMA form. Recursive least square system identification can be written for higher order systems (systems with more "active" moving parts) by increasing poles of the transfer function or simply increase the size matrices A and B. Matrix A is an adaptive system's state and B is the input matrix. There are C and D matrices as well which derive the output from state and input (feedforward), this where state space compensator or this case, the signal variance and residue. It can be used as an adaptive system calibrator for variety of robotic systems, image registration or pose estimation refinement.

	\item OCRTOC: Jerry helped me setup OCRTOC docker on my computer and showed me how to use it. I will continue with development on \cite{Dope} and \cite{MoreFusion} code bases. Next, I will try to generate data for training our model and catch up to Jerry's work. On the side, I will continue learning and try new ideas.,

	\item Reading list: \cite{lampinen2001bayesian} and \cite{li2019survey}.

	\item MoreFusion \cite{MoreFusion}: Still need to write a literature review on this.
	\item Broader Impact Project with Nolan: I pulled out my EE notes, I have a well organized set of notes from my undergrad studies. I took these notes with intend to one day publish them. The content is mostly ready, maybe need to add more text explanation. I believe you have seen my notes for computer vision, they all look exactly like that. I need to create a website and upload these notes as is or have them typed somehow. At some point this weekend, I will record at least one previous lecture and upload it to my YouTube channel and appropriate folder in lab OneDrive.

	\item UR5e:

	\item Quaternions:

	\item TensorFlow:

	\item Fellowship:

	\item Machine Learning: Working on Keras tutorial.




\end{itemize}

\newpage

\section{Plans}
Following items are listed in order of priority:

\begin{itemize}

	\item (On pause) Continue with ROS Industrial tutorials and documentation.

	\item (On pause) Resume Robotic Perception course as soon as possible.

	\item (On pause) Read Digital Image Processing by Gonzalez and Woods.

\end{itemize}



%Sets the bibliography style to UNSRT and import the
\newpage
\bibliography{bib_BardiaMojra_dd-mm2020}
\bibliographystyle{ieeetr}

\end{document}
