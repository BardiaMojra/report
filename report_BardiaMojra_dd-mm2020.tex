\documentclass[11pt]{article}
\usepackage{algorithm}
\usepackage{algpseudocode}
\usepackage{amsfonts}
\usepackage{amsmath}
\usepackage{amssymb}
\usepackage{amsthm}
\usepackage{bm}
\usepackage{color}
\usepackage{comment}
\usepackage{float}
\usepackage{graphicx}
\usepackage[hidelinks]{hyperref}
\usepackage{makecell}
\usepackage[caption=false,font=footnotesize,subrefformat=parens,labelformat=parens]{subfig}
\usepackage{wrapfig}
\usepackage{url}
\usepackage[table]{xcolor}
%
\setlength{\parindent}{0.25in}
\setlength{\parskip}{.05in}
\pagestyle{plain}
%Title, date an author of the document
\title{Progress Report}
\author{Bardia Mojra}


\begin{document}
\maketitle
\thispagestyle{empty}

\begin{center}
	\bigskip
	\bigskip
	Robotic Vision Lab

	University of Texas at Arlington
\end{center}

\newpage

\section{Progress}
Following items are listed in order of priority:
\begin{itemize}


	\item OCRTOC: This week Jerry and I worked on implementing a pose estimation solution for OCRTOC competition, we looked at PoseCNN \cite{PoseCNN}, DenseFusion \cite{DenseFusion}, DOPE \cite{NVlabs_2020_dope}. As discussed, DOPE is the most promising solution and Jerry is leading that effort. I started back on ROS industrial tutorials, it introduces a lot of features that I still need to get comfortable with. I also believe we might be able to draw inspiration from the latest kitting implementations found on GitHub.

	\item DenseFusion \cite{DenseFusion}: DenseFusion is a model for estimating 6D position for a given RGB-Depth input. First, it generates object segmentation masks and bounding boxes from RGB images. Moreover, it generate point cloud based on processed masked depth images. Then, RGB image crops are used to generate color embeddings using a CNN. In parallel, masked point clouds are used to generate geometry embeddings. The two embeddings are then pixel-wise dense fused together as one tensor for each training sample. Then a MLP is trained PointNet with Average Pooling, instead of Max Pooling, to encode the stochastic features or "information about the vicinity of each point and of the point cloud as a whole". Then it is concatenated to the tensor so that pixel-wise features are preserved.

	\item UR5: I began looking at installing Robotiq parallel gripper, it should be fairly straightforwards and simple. We still need to secure UR5e wires to the floor and table.

	\item Begin working on quaternions tutorial.

	\item TensorFlow: I went through chapter 1 TensorFlow implementation provided in \cite{CVwithTF2}. I need to begin using Anaconda to manage my Python workspaces. Nolan and Zongyao will assist me with getting started on that.

	\item Fellowship: I still need to work on my applications, I will write a new draft by next week. I have been avoiding this for far too long.

	\item Machine Learning: I did not work on ML course this week.




\end{itemize}

\newpage

\section{Plans}
Following items are listed in order of priority:

\begin{itemize}
	\item Continue to work on OCRTOC with Chris and begin working with instance segmentation ROS node.

	\item Continue with ROS Industrial tutorials and documentation.

	\item (On pause) Resume Robotic Perception course as soon as possible.

	\item (On pause) Read Digital Image Processing by Gonzalez and Woods.

\end{itemize}



%Sets the bibliography style to UNSRT and imports the
\newpage
\bibliography{bib_BardiaMojra_dd-mm2020}
\bibliographystyle{ieeetr}

\end{document}
