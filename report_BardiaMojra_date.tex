\documentclass[11pt]{article}
%\usepackage{algorithm}
%\usepackage{algpseudocode}
\usepackage{amsfonts}
\usepackage{amsmath}
\usepackage{amssymb}
\usepackage{amsthm}
\usepackage{bm}
\usepackage{color}
%\usepackage{comment}
%\usepackage{float}
%\usepackage{graphicx}
%\usepackage[hidelinks]{hyperref}
%\usepackage{makecell}
%\usepackage[caption=false,font=footnotesize,subrefformat=parens,labelformat=parens]{subfig}
%\usepackage{wrapfig} 
%\usepackage{url} 
%\usepackage[table]{xcolor} 
%
\setlength{\parindent}{0.25in}
\setlength{\parskip}{.05in}
\pagestyle{plain}
%Title, date an author of the document
\title{Progress Report}
\author{Bardia Mojra}


\begin{document}
	\maketitle
	\thispagestyle{empty}

\begin{center}
\bigskip
\bigskip
Robotic Vision Lab
 
University of Texas at Arlington
\end{center}

\newpage 

\section{Progress}
Following items are listed in order of priority: 
\begin{itemize}  

  	\item Fellowship: I still need to work on my applications, I will write a new draft by next week. I have been avoiding this for far too long. 
  	
  	\item Machine Learning: I did not work on ML course this week. I looked into using python toolchain with the help of hands-on books, I think I know what . I know what books I need, I considered buying them but it will be expensive and we already have some of them at the lab.  I think I would need a tablet instead, most of the books and Jupyter Notebooks are available online for free. 
  	
  	\item ARIAC: I did the tutorial for Gazebo Plugins and continued to setup ARIAC environment and realized it is best if we use a container such as Docker. Without it, it would be very difficult to setup and maintain Agile software development environment for multiple users on different machines. The reason is that for Catkin workspace to work properly, it needs to be setup in /tmp directory. Since Fedora, creation of /tmp directory has been permitted which is needed by communication software (i.e. ROS - Catkin) that need large global writeable address space. Unix shell assigns vertual address to these /tmp subdirectories and are routinely removed. This is why using a container makes sense, an entire workspace with local bash shell scripts, installed programs and configurations, and data and metadata log. It can be easily tracked for development, testing and deployment purposes; as well as, hardware-abstracted Agile software development. It makes long term software development much more feasible, specially with this level of sophistication.
    
    \item Gazebo: I finished plugin tutorial.
    
  	\item ROS: I continued with few tutorials on intermediate level. 
  
  	\item (On pause) Nolan and I have continued to work on TISR paper, I started a new document where I summarize background information. A copy of this document has been uploaded to GitHub.
 
  	\item (On pause) I still need to dissect \cite{PanopticSeg2019}, \cite{SVO}, \cite{HornsMethod}, \cite{NYUV2}, \cite{DGCNNLPC}, and \cite{MaskRCNN}.
  	
  	\item (On pause) I still need to dissect \cite{ZinsserWilliamKnowlton2006Oww}. 
\end{itemize}

\newpage 

\section{Plans}
Following items are listed in order of priority: 

\begin{itemize} 

	\item Go through ROS Industrial tutorials and documentation. 
	
	\item Go through UR Gazebo documentation. 

  	\item Resume Robotic Perception course as soon as possible.  
  
  	\item (On pause) Begin working on quaternions tutorial.  
 
  	\item (On pause) Need to read \cite{ImSRwDeepCNN}, \cite{MixDNNforSISR}, \cite{mModalSemanticSLAMwProb}, and \cite{RCANforImClass}; these papers seem fundamental to understanding the overall picture.

  	\item (On pause) Read Digital Image Processing by Gonzalez and Woods. 
 
 \end{itemize}



%Sets the bibliography style to UNSRT and imports the 
\newpage 
\bibliography{report_BM_bib} 
\bibliographystyle{ieeetr}

\end{document}
