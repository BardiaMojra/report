\documentclass[11pt]{article}
\usepackage{bookmark}
\usepackage{algorithm}
\usepackage{algpseudocode}
\usepackage{amsfonts}
\usepackage{amsmath}
\usepackage{amssymb}
\usepackage{amsthm}
\usepackage{bm}
\usepackage{color}
\usepackage{comment}
\usepackage{float}
\usepackage{graphicx}
%\usepackage[hidelinks]{hyperref}
\usepackage{makecell}
\usepackage[caption=false,font=footnotesize,subrefformat=parens,labelformat=parens]{subfig}
\usepackage{wrapfig}
\usepackage{url}
\usepackage[table]{xcolor}
%
\setlength{\parindent}{0.25in}
\setlength{\parskip}{.05in}
\pagestyle{plain}
%Title, date an author of the document
\title{Progress Report}
\author{Bardia Mojra}


\begin{document}
\maketitle
\thispagestyle{empty}

\bigskip
\bigskip
\begin{center}
      Robotic Vision Lab
\end{center}

\begin{center}
      The University of Texas at Arlington
\end{center}

\newpage

\section{Specific Research Goals}
\begin{itemize}
      \item VPQEKF (IROS - Mar. 1st): Work on the paper.
      \item DLO Manipulation Proposal: Work on a personal statement.
\end{itemize}

\section{To Do}
\begin{itemize}
  \item Fellowship: Need finish my personal statement.
  \begin{itemize}
      \item EERE (DoE-AMO) - 1/25/22
      \item Maverick Merit - 2/11/22
      \item TACC - 2/18/22
      \item Maverick Doctoral Bridge - 4/29/22
  \end{itemize}
  \item  DLO Manipulation:
  \begin{itemize}
      \item Write paragraphs on separate ideas and edit later
      \item Real dataset
      \begin{itemize}
            \item Design, discuss and build a data collection and test rig
            \item Define DLO classes and specs
            \item Purchase DLO samples for data collection
      \end{itemize}
      \item Unity dataset
      \begin{itemize}
            \item Recreate virtual duplicates of physical test material
            \item Model dynamics and deformity
  \end{itemize}
  \item PVQEKF (Paper deadline March 1st.):
  \begin{itemize}
      \item Finish Dead Reckoning code -- ASAP
      \item DON'T DO PVQEKF WITH ROS, SKIP FOR NOW, USE PYTHON
      \item Setup ROS environment -- (1) --
      \item Replace EKF with QEKF -- (2) --
      \item Feature point extraction:
      \item Depth to scale
      \item BigC (where we solve Q+V together) --> regarding depth scale issue
      \item Quat: switching problem is fixed
      \item 35 solutions (start here)
      \item Noise issue: noise cannot be modelled
      \item Chaining step: when feature points come in and out of the frame dependency configuration.
  \end{itemize}
\end{itemize}


\section{Progress}
The following items are listed in the order of priority:
\begin{itemize}
      \item Fellowship: I updated my resume this week and familiarized myself
      with EERE application process. Other opportunities, i.e. summer internships
      and departmental fellowships are under consideration. They are EERE summer
      internship (DoE-AMO-1/25/22), Maverick Merit fellowship (2/11/22), TACC
      fellowship (2/18/22), and Maverick Doctoral Bridge fellowship (4/29/22),
      with deadlines stated in parenthesis. EERE is initiated and managed by
      Automation Management Office (AMO), as a part Department of Energy (DoE).
      This opportunity seems most in line with my research and I will focus on
      applying this weekend. For the Maverick Merit fellowship; essentially,
      I need to convince the department to nominate me because my 'academic
      preparation and accomplishments exceed attainments of students who the
      program typically admits unconditionally by a significant margin." I
      believe I might have a chance but I don't have any publications yet. What
      do you think?
      \item Dead Reckoning (March 1st, 2022): I worked on preprocessing raw
      sensor data collected from an android device. I learned about Unix epoch
      format and used Matlab to resample interpolated measurements to create a
      unified-time dataset. Currently, I am working on state estimation (QEKF)
      part of the project by trying to integrate acceleration and gyro data.
      \item VPQEKF (March 1st, 2022): I need to start working on feature points
      extraction done by Quest+Vest code \cite{quest}.
      \item DLO Manipulation: I need to create a more developed Gantt chart,
      routinely brainstorm new ideas, and dedicate time to the progress of the
      project. Below is a brief initial task list:
      \begin{itemize}
            \item Write paragraphs on separate ideas and edit later
            \item Real dataset:
            \begin{itemize}
                  \item Design, discuss and build a data collection and test rig
                  \item Define DLO classes and specs
                  \item Purchase DLO samples for data collection
                  \item Create data collection pipeline with capture, preprocessing,
                  annotation, and storage modules.
                  \item Develop a series of easy-to-perform standard dynamic tests
                  for system identification of DLO's.
            \end{itemize}
            \item Unity dataset
            \begin{itemize}
                  \item Recreate virtual duplicates of physical test material
                  \item Model dynamics and deformity
            \end{itemize}
            \item Develop "Optimal Real-Time Dynamic Model and Control" theory:
            \begin{itemize}
                  \item Definition: Define what is to be optimized, over what domain,
                  and with respect to what parameters and constraints. A robust
                  ranking system or method is needed for almost all parameters.
                  \item Define "Unit Segment" for deformable objects: define clearly
                  with literature review.
                  \item Estimated Parameters: estimated parameters of the object of
                  interest which may \textit{slowly} change over time, permanently
                  or temporarily' i.e. DLO stiffness due to repetitive bending or
                  ambient temperature.
                  \item Estimated Features: These are estimated constant features of
                  the objects that do not change over time i.e. estimated length of a
                  relatively short DLO in an environment with stable ambient
                  temperature.
                  \item Fundamental Features: These are the known characteristic
                  parameters of a DLO that in some cases would be given to the system.
                  These fundamental factors will not
                  change over time, i.e. length of the DLO, cross-section thickness,
                  and other known constant physical and dynamic characteristic
                  features. If one fundamental parameter of an object changes (with
                  specifics to be defined), we will treat it as a new object.
                  \item Limiting Factors: These would be imposed or assumed constraints
                  that limit the control state solution space i.e. unstable and
                  impossible configurations, self-occlusion, and controller real-time
                  compute time and real-time performance metric (for real-time and
                  online self-evaluation).
                  \item Other constraints such as external forces, lighting, DLO
                  variations.
                  \item Online Control Performance Self-Evaluation: This is needed for
                  accurate evaluation and optimization of control performance in
                  real-time.
            \end{itemize}
      \end{itemize}
      \item Problem setup and reformulation: closed-loop geometric control
            system for real-time and stable feedback of the physical system.
      \item Optimization of dynamic system model for the most accurate
            real-time state estimation, control, and object manipulation.
      \item DLO parameter initialization on first sight: The nonlinear
            nature of DNN's could be exploited to train an adaptive model that
            predicts object parameters from a single image. Such quick estimation
            of the system's characteristic parameters (i.e. unknown spring and
            damping constants) could enable quick and precise control of DLO's.
            Moreover, these parameters will be tuned online using
            adaptive learning and control.
      \item NBV-Grasping Project: No update.
      \item PyTorch Tutorials: Transfer learning.
      \item Pose Estimation: I will need it for DLO segment localization.
\end{itemize}

%\newpage

\section{Intermediate Goals - Fall 2021:}
\begin{itemize}
      \item QEKF: Finish paper.
      \item Active Learning.
      \item UR5e: Do the tutorials.
\end{itemize}


%Sets the bibliography style to UNSRT and import the
\newpage
\bibliography{ref}
\bibliographystyle{ieeetr}

\end{document}
