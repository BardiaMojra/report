\documentclass[11pt]{article}
\usepackage{bookmark}
\usepackage{algorithm}
\usepackage{algpseudocode}
\usepackage{amsfonts}
\usepackage{amsmath}
\usepackage{amssymb}
\usepackage{amsthm}
\usepackage{bm}
\usepackage{color}
\usepackage{comment}
\usepackage{float}
\usepackage{graphicx}
%\usepackage[hidelinks]{hyperref}
\usepackage{makecell}
\usepackage[caption=false,font=footnotesize,subrefformat=parens,labelformat=parens]{subfig}
\usepackage{wrapfig}
\usepackage{url}
\usepackage[table]{xcolor}
\graphicspath{{images_dd-mm2022/}}
\setlength{\parindent}{0.25in}
\setlength{\parskip}{.05in}
\pagestyle{plain}
%Title, date an author of the document
\title{Progress Report}
\author{Bardia Mojra}


\begin{document}
\maketitle
\thispagestyle{empty}

\bigskip
\bigskip
\begin{center}
 Robotic Vision Lab
\end{center}

\begin{center}
The University of Texas at Arlington
\end{center}

\newpage

\section{Specific Research Goals}
\begin{itemize}
      \item VPQEKF (\textcolor{red}{---}): Work on the paper.
      \item DLO Manipulation Dataset (ICRA - \textcolor{red}{Sept. 1st})
\end{itemize}

\section{To Do}
\begin{itemize}
  \item QEKF Paper - 30\% extension (\textcolor{red}{---}): QuEst solutions!!
  \item Implementation (\textcolor{red}{---}):
  \begin{itemize}
      \item Noise issue: noise cannot be modeled - revisit
      \item SfM: RQuEst cannot find solution -- under investigation - HAVOK?
  \end{itemize}
  \item  DLO Manipulation: (\textcolor{red}{ICRA - Sept. 1st})
  \begin{itemize}
      \item Work on the paper everyday -- up-coming
      \item ICRA 2022 RL workshops: gym, stable-baseline3, and RL zoo -- on-going
      \item Setup digital twin reinforcement learing setup:
      \begin{itemize}
        \item Unity Robotics extension setup -- on-going.
        \item Design dynamic DLO data collection system.
        \item Build work cell. -- on-going
        \item Collect data and create a dataset.
        \item Define evaluation metrics.
        \item Create a high frequency RGBD dataset with UV-frames and open-loop input control actions as the ground truth.
      \end{itemize}
      \item Real-Time Preception -- on hold
      \item Learning DLO Dynamics and System Identification
      \begin{itemize}
            \item List feasible approached for learing DLO dynamics -- done
            \item Model dynamics and deformity in a latent space
      \end{itemize}
  \end{itemize}
\end{itemize}

\section{Progress}
The following items are listed in the order of priority:
\begin{itemize}
    \item XEst (\textcolor{red}{RAL ---}): No technical update. I have an
    idea for finding the correct Quaternion solution at the end of QuEst.
    Instead of trying to find the best 5 point correspondacnes and running a
    low dimensional linear decomposition, we could keep all the point
    correspondences and run Dynamic Mode Decomposition (DMD)
    \cite{schmid2010dynamic} \cite{kutz2016dynamic} to dilute
    the effect of i.i.d. noise on image pixels. Another method could be HAVOK
    \cite{brunton2017chaos} which uses DMD and Koopman operator to estimate
    a low dimensional linear estimate of the transform. The distribution of
    intermitted forcing in HAVOK could be an indication of how much noise we
    have bundled with selected point correspondacnes.


    \item DLO State Estimation (\textcolor{red}{ICRA - Sept. 1st}): This week,
    I researched Koopman operator, dynamic mode decomposition (DMD)
    \cite{kutz2016dynamic} \cite{schmid2010dynamic},
    extended dynamic mode decomposition (EDMD) \cite{williams2015data},
    Hankel alternative view of Koopman (HAVOK) \cite{brunton2017chaos}, and
    sparse identification of nonlinear dynamic systems (SiNDy) \cite{brunton2016discovering}.
    All these methods are very similar in principle as they combine principal
    component analysis (PCA) with frequency analysis to represent system or data
    dynamics with respect to both time and space. Each of the mentioned methods
    slightly differ from each other. For example, HAVOK and Koopman are low
    dimensional where DMD and EDMD are high dimensional methods. Koopman and
    HAVOK are continuous where SinDy is sparse. I will test all these methods
    and present the results in my DLO paper. I will expand more in the paper.

    Moreover, I finished setting up Refind boot manager and installed Ubuntu 18.04.
    My plan is to use the simulation environment provided in \cite{yu2022shape}
    to generate appropriate dataset for our experiments.

    \item Maicol (REU): He has finals and he is busy doing Unity tutorials.
    \item PyTorch Tutorials: Transfer learning.
    \item Omniverse: Apply for access. -- To-Do

  \end{itemize}


\newpage

%Sets the bibliography style to UNSRT and import the
\newpage
\bibliography{ref}
\bibliographystyle{ieeetr}

\end{document}
