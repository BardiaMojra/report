\documentclass[11pt]{article}
\usepackage{bookmark}
\usepackage{algorithm}
\usepackage{algpseudocode}
\usepackage{amsfonts}
\usepackage{amsmath}
\usepackage{amssymb}
\usepackage{amsthm}
\usepackage{bm}
\usepackage{color}
\usepackage{comment}
\usepackage{float}
\usepackage{graphicx}
%\usepackage[hidelinks]{hyperref}
\usepackage{makecell}
\usepackage[caption=false,font=footnotesize,subrefformat=parens,labelformat=parens]{subfig}
\usepackage{wrapfig}
\usepackage{url}
\usepackage[table]{xcolor}
\graphicspath{{images/}}
\setlength{\parindent}{0.25in}
\setlength{\parskip}{.05in}
\pagestyle{plain}
%Title, date an author of the document
\title{Progress Report}
\author{Bardia Mojra}


\begin{document}
\maketitle
\thispagestyle{empty}

\bigskip
\bigskip
\begin{center}
 Robotic Vision Lab
\end{center}

\begin{center}
The University of Texas at Arlington
\end{center}

\newpage

\section{Specific Research Goals}
\begin{itemize}
      \item VPQEKF (\textcolor{red}{April 13th}): Work on the paper.
      \item DLO Manipulation Dataset (ICRA - \textcolor{red}{Sept. 1st})
\end{itemize}

\section{To Do}
\begin{itemize}
  \item QEKF Paper - 30\% extension (\textcolor{red}{April 13th}):
  \begin{itemize}
      \item Edit VEst section and add updates.
  \end{itemize}
  \item QEKF/QuEst+VEst Implementation (\textcolor{red}{April 13th}):
  \begin{itemize}
      \item OOP Integration: QuEst is tested and other algorithms have been
      integrated and tested as well. Integrate VEst --- On-going.
      \item Feature point extraction: implement semantic segmentation
      \item Address scale factor (depth-scale) issues: DL solutions?
      \item Address "hand off" issue when objects enter or leave field of view
      \item Real-time streaming images for real-time operation (optional)
      \item Experiments
      \item Noise issue: noise cannot be modeled - revisit
  \end{itemize}
  \item  DLO Manipulation:  \textcolor{red}{Sept. 1st}
  \begin{itemize}
      \item Find other ICRA dataset papers and summarize the structure. --- This week.
      \item Dataset (ICRA -  \textcolor{red}{Sept. 1st}):
      \begin{itemize}
            \item Finalize MoCap design, design digital twin work cell. --- This week.
            \item Build work cell.
            \item Collect data and create a dataset.
            \item Create object dynamics ground-truth method, format, and evaluation
            metrics.
      \end{itemize}
      \item Control and Tracking
      \begin{itemize}
            \item Create UR5+DLO simulation in Matlab and begin work on H-Infinity control before Reza leaves for Indiana State.
            \item Model dynamics and deformity
      \end{itemize}
      \item Real-Time Preception
      \begin{itemize}
        \item Implement PVnet, perform transfer learning and retrain using
        in house dataset.
        \item Time model inference, using auto-encoders generate the lowest
        dimensional representation for each object.
        \item Use another GAN model for object deformity for each object.
        \item Evaluate encoded representation for accuracy.
        \item Used another GAN to explore other abstraced representations from
        individual encoded representation. In theory, we can create a low
        dimensionsal representation for multiple similar objects, given all
        individual low-dimensional representations. This is inspired by "fundamental
        principles first" approach which has universal applicability.
      \end{itemize}
      \item
  \end{itemize}
\end{itemize}


\section{Progress}
The following items are listed in the order of priority:
\begin{itemize}
    \item VPQEKF (\textcolor{red}{RAL - April 1st, 2022}): This week, I finished
    testing and validating QuEst source code into the integrated system
    environment. Moreover, I integrated other classical pose estimation algorithms.

    \begin{figure}
      \includegraphics[width=\linewidth]{pose_est_algs.png}
      \caption{Classical Pose Estimation Methods}
      \label{fig:poses}
    \end{figure}

    \item DLO Dataset: I will look into the bill of material and the final design
    this week. I should perhaps have a discussion with Sami about this. I am sure
    he knows a lot more than me. He can probably provide very good insight on
    topology representation and preception. Moreover, I took the following photos
    as initial samples. In my work, I need to take more photos of the marker
    from different angles and train a minimal model that robustly detects this
    marker from any angle (with glare). I have not gone through the papers we
    found last week. In fact, it seems that I have lost them since I had to reinstall
    my OS on Friday. I will make an outline for the dataset paper and will do
    some tutorials on Dataset generation from ICRA.

    \begin{figure}
      \begin{center}
      \includegraphics[width=.6\linewidth]{DLO_0001.png}
      \caption{DLO Sample Image}
      \label{fig:DLO01}
      \end{center}
    \end{figure}

    \begin{figure}
      \begin{center}
      \includegraphics[width=.6\linewidth]{shiny_checks.png}
      \caption{Shiny Marker}
      \label{fig:Marker01}
      \end{center}
    \end{figure}

    \item DLO Control: I had a meeting with Dr. Gans and Reza
    where we discussed H-Infinity and Mu-Synthesis control design and it seems
    to be a great fit for controlling a DLO in stable congfiguration. In theory,
    a H-infinity controller takes advantage of one additional constraint to bound
    the output. The constaint is conservation of energy from thermodynamics.
    In context of H-infinity control, it assumes that no matter how unstable the
    output, its magnitude will never be larger than the energy of the input and
    the system itself (maximum gain in any dimension and at any frequency). The controller pays attention
    to the ratio of the input and resulting output change. Thus,
    one can define a optimization criterion, e.g. local estimation accuracy,
    model input noise to the system and H-infinity controller will guarantee
    the system will preform within a certain bound. This bound is defined by
    ratio of magnitude of the input and the system where sensitivity is minimized
    recursively. This creates robustness in control while maintaining optimality
    to defined criterion. H-infinity is used for linear or linearized systems and
    Mu-synthesis is for nonlinear systems where there is uncertainty. Moreover,
    Reza mentioned \emph{Linear Parameter Varying control} (LPV) where a nonlinear
    system is model as parameterized linear system with \emph{Gain Scheduling}.
    LPV can be particularly helpful in controlling DLOs, and perhaps it could
    be extended to multiple DLO classes. We can begin this project in simulation
    and quickly publish a paper with Reza on just the controller design for DLOs.

    \item DLO Perception: Building scanner would be too time consuming. Ensenso
    makes high quality 3D scanners and UTARI has one. We could borrow it or we can
    set the scanner there \cite{3DCamera52:online}.

    \item Semantic segmentation (\textcolor{blue}{DLO-02}): Per my discussion with Dr. Gans, I will explore DL methods for the depth or scale problem.
    \item Grasping Project (\textcolor{blue}{DLO-03}): I am making this a part of the DLO project.
    \item PyTorch Tutorials: Transfer learning.

  \end{itemize}

\section{Intermediate Goals - Fall 2021:}
\begin{itemize}
      \item QEKF: Finish paper.
      \item UR5e: Do the tutorials.
\end{itemize}

\newpage

%Sets the bibliography style to UNSRT and import the
\newpage
\bibliography{ref}
\bibliographystyle{ieeetr}

\end{document}
