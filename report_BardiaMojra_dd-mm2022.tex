\documentclass[11pt]{article}
\usepackage{bookmark}
\usepackage{algorithm}
\usepackage{algpseudocode}
\usepackage{amsfonts}
\usepackage{amsmath}
\usepackage{amssymb}
\usepackage{amsthm}
\usepackage{bm}
\usepackage{color}
\usepackage{comment}
\usepackage{float}
\usepackage{graphicx}
%\usepackage[hidelinks]{hyperref}
\usepackage{makecell}
\usepackage[caption=false,font=footnotesize,subrefformat=parens,labelformat=parens]{subfig}
\usepackage{wrapfig}
\usepackage{url}
\usepackage[table]{xcolor}
\graphicspath{{images_dd-mm2022/}}
\setlength{\parindent}{0.25in}
\setlength{\parskip}{.05in}
\pagestyle{plain}
%Title, date an author of the document
\title{Progress Report}
\author{Bardia Mojra}


\begin{document}
\maketitle
\thispagestyle{empty}

\bigskip
\bigskip
\begin{center}
 Robotic Vision Lab
\end{center}

\begin{center}
The University of Texas at Arlington
\end{center}

\newpage

\section{Specific Research Goals}
\begin{itemize}
      \item VPQEKF (\textcolor{red}{May 30th}): Work on the paper.
      \item DLO Manipulation Dataset (ICRA - \textcolor{red}{Sept. 1st})
\end{itemize}

\section{To Do}
\begin{itemize}
  \item QEKF Paper - 30\% extension (\textcolor{red}{June 30th}):
  \item Implementation (\textcolor{red}{May 30th}):
  \begin{itemize}
      \item Point-feature extraction: tracking issue -- resolved
      \item Address scale factor (depth-scale) issues: -- resolved
      \item Noise issue: noise cannot be modeled - revisit
      \item Adding plots -- done
      \item Rewrite RQuEst (RANSAC-QuEst) to use triangulation
      \item Replace the relative pose estimation routine in SfM example with RQuEst
  \end{itemize}
  \item  DLO Manipulation:  \textcolor{red}{ICRA - Sept. 1st}
  \begin{itemize}
      \item Work on the paper everyday -- up-coming
      \item Find other ICRA dataset papers and summarize the structure. -- Done.
      \item Setup Omniverse on TACC -- Next
      \item Setup digital twin reinforcement learing setup:
      \begin{itemize}
        \item Design dynamic DLO data collection system.
        \item Build work cell.
        \item Collect data and create a dataset.
        \item Define evaluation metrics.
        \item Create a high frequency RGBD dataset with UV-frames and open-loop input control actions as the ground truth.
      \end{itemize}
      \item Real-Time Preception
      \begin{itemize}
        \item Deep learning methods for keypoint pose estimation in real-time.
        \item Use UV dye dataset
        \item Use PVNet-like approach for known-object pose estimation.
      \end{itemize}
      \item Learning DLO Dynamics and System Identification
      \begin{itemize}
            \item List feasible approached for learing DLO dynamics
            \item Model dynamics and deformity in a latent space
      \end{itemize}
      \item Real-Time Control
      \begin{itemize}
        \item Time model inference, using auto-encoders generate the lowest
        dimensional representation for each object.
        \item Use another GAN model for object deformity for each object.
        \item Evaluate encoded representation for accuracy.
        \item Used another GAN to explore other abstraced representations from
        individual encoded representation. In theory, we can create a low
        dimensionsal representation for multiple similar objects, given all
        individual low-dimensional representations. This is inspired by "fundamental
        principles first" approach which has universal applicability.
      \end{itemize}
  \end{itemize}
\end{itemize}


\section{Progress}
The following items are listed in the order of priority:
\begin{itemize}
    \item XEst (\textcolor{red}{RAL - April 30st, 2022}): I finished working on
    the plotting module for QEKF and was able to generate log plots for each of
    the pose estimation algorithms. Although, there is an issue with the
    quaternion estimation output where it does not update properly. It seems to
    be a tuning issue, I looked into it but the issue presisted and I am working
    Bundle Adjustment so I shelved QEKF for now.



    \begin{figure}[H]
      \begin{center}
        \includegraphics[width=\linewidth]{fig_mat-feat_kf00004.png}
      \end{center}
      \caption{Acceptable point-feature matches between two frames.}
    \end{figure}



    \begin{figure}[H]
      \begin{center}
        \includegraphics[width=\linewidth]{fig_mat-feat_kf00022.png}
      \end{center}
      \caption{Noisy point-feature matches between two frames.}
    \end{figure}

    \begin{figure}[H]
      \begin{center}
        \includegraphics[width=.8\linewidth]{line_features.png}
      \end{center}
      \caption{Line-features from an image sequence.}
    \end{figure}

    \begin{figure}[H]
      \begin{center}
        \includegraphics[width=\linewidth]{plt_pos_log_Nister.png}
      \end{center}
      \caption{Pose estimation log: Nister vs. Groundtruth.}
    \end{figure}


    \begin{figure}[H]
      \begin{center}
        \includegraphics[width=\linewidth]{plt_pos_log_QuEst.png}
      \end{center}
      \caption{Pose estimation log: QuEst vs. Groundtruth.}
    \end{figure}

    \item XEst - Semantic segmentation (\textcolor{red}{RAL - April 30st, 2022}):
    No update on implementing \cite{ballester2021dot}.

    \item DLO Dataset: I installed Unity and soon will begin working on tutorials.
    \item Linus (REU): He is working on Unity tutorials, recreating RVL workcell,
    and importing UR5 model into Unity.
    \item Maicol (REU): He is working on ROS2 tutorials, MoveIt tutorials, and
    way-point navigation of UR5 in Unity.
    \item Myself (with REU): I will start on MuJuCo tutorials as well.

    \item DLO Control (MuJuCo): No update.

    \item Grasping Project (\textcolor{blue}{DLO-03}): I am making this a part of the DLO project.
    \item PyTorch Tutorials: Transfer learning.
    \item Manifold learning: Marcus emailed me some papers, I will read them
    and reply to him. I am not particularly interested in the project but his
    ideas are interesting and I would like to help him if I can. He is very
    knowledgeable on mathematics and I cherish that.

  \end{itemize}

\section{Intermediate Goals - Fall 2021:}
\begin{itemize}
      \item QEKF: Finish paper.
      \item UR5e: Do the tutorials.
\end{itemize}

\newpage

%Sets the bibliography style to UNSRT and import the
\newpage
\bibliography{ref}
\bibliographystyle{ieeetr}

\end{document}
