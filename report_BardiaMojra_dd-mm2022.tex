\documentclass[11pt]{article}
\usepackage{bookmark}
\usepackage{algorithm}
\usepackage{algpseudocode}
\usepackage{amsfonts}
\usepackage{amsmath}
\usepackage{amssymb}
\usepackage{amsthm}
\usepackage{bm}
\usepackage{color}
\usepackage{comment}
\usepackage{float}
\usepackage{graphicx}
%\usepackage[hidelinks]{hyperref}
\usepackage{makecell}
\usepackage[caption=false,font=footnotesize,subrefformat=parens,labelformat=parens]{subfig}
\usepackage{wrapfig}
\usepackage{url}
\usepackage[table]{xcolor}
\graphicspath{{images/}}
\setlength{\parindent}{0.25in}
\setlength{\parskip}{.05in}
\pagestyle{plain}
%Title, date an author of the document
\title{Progress Report}
\author{Bardia Mojra}


\begin{document}
\maketitle
\thispagestyle{empty}

\bigskip
\bigskip
\begin{center}
 Robotic Vision Lab
\end{center}

\begin{center}
The University of Texas at Arlington
\end{center}

\newpage

\section{Specific Research Goals}
\begin{itemize}
      \item VPQEKF (April 1st): Work on the paper.
      \item DLO Manipulation Dataset (September - ICRA or IROS?)
\end{itemize}

\section{To Do}
\begin{itemize}
  \item QEKF Paper - 30\% extension (April 1st):
  \begin{itemize}
      \item Edit VEst section and add updates.
  \end{itemize}
  \item QEKF/QuEst+VEst Implementation (\textcolor{red}{Feb. 28th}):
  \begin{itemize}
      \item Implement QuEst 5-point - On-going.
      \item Implement VEst
      \item Address scale factor (depth-scale) issues
      \item Address "hand off" issue when objects enter or leave field of view
      \item Real-time streaming images for real-time operation (optional)
      \item Experiments
      \item Feature point extraction - On-going.
      \item Noise issue: noise cannot be modeled
  \end{itemize}
  \item  DLO Manipulation:
  \begin{itemize}
      \item Related work literature review
      \item Real dataset + paper (September 2022 - ICRA):
      \begin{itemize}
            \item Watch IROS manipulation workshop videos. - Done.
            \item Design, discuss and build a data collection and test rig.
      \end{itemize}
      \item Unity dataset
      \begin{itemize}
            \item Recreate virtual duplicates of physical test material
            \item Model dynamics and deformity
      \end{itemize}
  \end{itemize}
\end{itemize}


\section{Progress}
The following items are listed in the order of priority:
\begin{itemize}
    \item VPQEKF (April 1st, 2022): The rewrite and debugging of the QuEst
    the algorithm in Python is finished but a few utility functions such as
    evaluation, data logging, and visualization remain unfinished. The new
    code has been compared against the original implementation line by line
    and so far everything has matched and with few improvements. There are
    three notable improvements thus far, 1) better feature matching and ranking
    of best-matched features between frames, 2) and more fine-tuned control
    over of variable data type in Python with Numpy Quaternion module. 3) Also,
    the Python implementation has resulted in fewer nonzero Quaternion solutions
    for equation 20 in \cite{quest}.\\
    In the new Python implementation, I took advantage of highly the developed
    OpenCV modules and used ORB feature detector instead of SIFT. The initial
    results have been very promising and ORB is supposed to have a shorter
    detection time than SIFT. I still need to time my implementation and
    compare it against what is presented in \cite{quest}. Moreover, for this
    implementation, I used Numpy Quaternion module that provides a fast and
    reliable Quaternion algebra library. By default, it declares all
    floating-point variables as \emph{float64} and it has a dedicated Quaternion data
    type, i.e. \emph{Quaternion}. In Numpy, a Quaternion is declared as a set
    of four \emph{float128} variables. In order to preserve full numerical
    accuracy, all data handlers are declared \emph{float128} at initialization
    instead of default data type, \emph{float64}. Additionally, the
    implementation repeatedly produces a smaller set of solutions compared to
    it Matlab implementation, 24 instead 32. This is beneficial since it shrinks
    the pool of possible solutions. The root cause of this is under
    investigation but it might be related to numerical accuracy and the
    computational rounding effect.
    For the evaluation part, we followed \cite{huynh2009metrics}
    where the author presented six functions for measuring the distance between 3D
    rotations. This is particularly trickly for Quaternions as they are
    normalized when representing rotation from the previous frame or relative
    orientation. The paper presents 5 metrics that are suitable for operation
    on \emph{SO(3)}, three of which I am currently implementing.\\
    Next, I will finish the implementation and begin writing on the paper this
    week. I know enough about the paper that I can write it from scratch with
    you and Dr. Gans as co-authors rather than doing an extension. I would
    appreciate your guidance on this.

    \item DLO Manipulation: I read the paper you assigned to me and thank you
    for doing that. At this point, I have had a similar picture in my mind
    regarding how the grasping task could be achieved. As described in the paper,
     the authors first localize cylindrical objects individually within the frame
      and mask everything else. Then, they proceed to estimate the grasping pose
       for each object. They use a mask to isolate each object and they provide RGB and depth as input for their deep learning model. They trained their model to estimate midpoint, angle, and 'probability of grasping point' which I assume refers to the inferred probability of ideal grasping pose. They follow up with an alignment step that technically should decrease the rate of falling items but the authors do not provide evidence of its effectiveness. It is a clever and simple solution to a relatively complicated problem and
    it could be useful in real-world situations, e.g. pick and place on a production line.
    The authors do not provide much information on current literature and related works. This completely undermines the authors' work as a similar or more advanced method could be already published. How can we identify any contributions without making comparisons to the state of the art? The authors did not compare their grasping performance against any other existing methods, nor do they mention whether such methods exist or not. On the merits of the submission, there is no scientific or academic contribution, nor do the authors make such claims. What is described is an engineered solution to the grasping task which in all fairness is a difficult one to implement. As mentioned by the authors, I think this is a work in progress and at this time and in this form, this work does not amount to a scientific contribution. Moreover, the authors claim their proposed method is capable of grasping unknown or never-seen-before objects. In my opinion, such characterization would be a stretch of imagination and play of words; it is referring to grasping cylindrical objects versus cylindrical batteries. In my opinion, they should be considered known objects as to the system they are simply, both cylindrical objects.

    \item Pose Estimation: I will need it for DLO segment localization.
    \item NBV-Grasping Project: I talked to Chris about his grasping project
    and he is not pursuing that project anymore. I will work on grasping after
    the QEKF project.
    \item PyTorch Tutorials: Transfer learning.
    \item SBD Funding: I was thinking about writing a short report for them to
    attach or to provide with our thank you message. In there, we could explain
    multiple applications where machine learning could be used to solve their
    problems. I know where they need robots, it would be in final-stage
    manufacturing automation and tool qualification test procedures and both
    are considered pick-and-place applications. Final-stage
    manufacturing automation would be a UR5 robotic arm moving parts from a
    tray to a conveyer belt or placing raw material in a CNC machine and
    removing the finished part. The tool qualification test procedures are
    highly dependent on the tool and mostly involve simulating physical use of
    the tool as a user would. Technicians test a large number of tools on
    various predetermined tests every time there are ANY changes made to the
    design of a tool. These cost a lot of money and the jobs are tedious as
    they have very low employee retention rates. The testing is a real
    bottleneck and we got yelled at if our code ever failed the test because
    it has to be fixed and tested from the beginning with new tools. They often
    test hundreds of the same tools at a time if big enough changes were
    made to its design. This might
    not seem much, but there are about four or five brands with 250 tools under
    each and the entire engineering operation is handled by ~3000 employees
    where only about 15 are embedded software engineers.
    They produce and sell millions of tools every month, quality is extremely
    important and every penny matters to them.
    If we can provide some demos and perhaps set up
    a test rig with two UR5s in Towson, I am willing to bet could ask for
    half to a million-dollar from them. I have seen them pay two million dollars
    for a useless BLE source code that my colleague ended up rewriting anyway.
    It is a wild world.
  \end{itemize}

\section{Intermediate Goals - Fall 2021:}
\begin{itemize}
      \item QEKF: Finish paper.
      \item UR5e: Do the tutorials.
\end{itemize}

\newpage

%Sets the bibliography style to UNSRT and import the
\newpage
\bibliography{ref}
\bibliographystyle{ieeetr}

\end{document}
