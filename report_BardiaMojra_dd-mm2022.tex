\documentclass[11pt]{article}
\usepackage{bookmark}
\usepackage{algorithm}
\usepackage{algpseudocode}
\usepackage{amsfonts}
\usepackage{amsmath}
\usepackage{amssymb}
\usepackage{amsthm}
\usepackage{bm}
\usepackage{color}
\usepackage{comment}
\usepackage{float}
\usepackage{graphicx}
%\usepackage[hidelinks]{hyperref}
\usepackage{makecell}
\usepackage[caption=false,font=footnotesize,subrefformat=parens,labelformat=parens]{subfig}
\usepackage{wrapfig}
\usepackage{url}
\usepackage[table]{xcolor}
\graphicspath{{images/}}
\setlength{\parindent}{0.25in}
\setlength{\parskip}{.05in}
\pagestyle{plain}
%Title, date an author of the document
\title{Progress Report}
\author{Bardia Mojra}


\begin{document}
\maketitle
\thispagestyle{empty}

\bigskip
\bigskip
\begin{center}
 Robotic Vision Lab
\end{center}

\begin{center}
The University of Texas at Arlington
\end{center}

\newpage

\section{Specific Research Goals}
\begin{itemize}
      \item VPQEKF (RAL - April 1st): Work on the paper.
      \item DLO Manipulation Dataset (ICRA - September)
\end{itemize}

\section{To Do}
\begin{itemize}
  \item QEKF Paper - 30\% extension (April 1st):
  \begin{itemize}
      \item Edit VEst section and add updates.
  \end{itemize}
  \item QEKF/QuEst+VEst Implementation (\textcolor{red}{Feb. 28th}):
  \begin{itemize}
      \item Implement QuEst 5-point: Done, debugging.
      \item Feature point extraction: implement semantic segmentation
      \item Implement VEst
      \item Address scale factor (depth-scale) issues: DL solutions?
      \item Address "hand off" issue when objects enter or leave field of view
      \item Real-time streaming images for real-time operation (optional)
      \item Experiments
      \item Noise issue: noise cannot be modeled
  \end{itemize}
  \item  DLO Manipulation:
  \begin{itemize}
      \item Related work literature review
      \item Real dataset + paper (September 2022 - ICRA):
      \begin{itemize}
            \item Design, discuss and build a data collection and test rig.
      \end{itemize}
      \item Unity dataset
      \begin{itemize}
            \item Recreate virtual duplicates of physical test material
            \item Model dynamics and deformity
      \end{itemize}
  \end{itemize}
\end{itemize}


\section{Progress}
The following items are listed in the order of priority:
\begin{itemize}
    \item VPQEKF (\textcolor{red}{RAL - April 1st, 2022}): This week, I finished
    working on the RANSAC QuEst module. Initially, I was following Kaveh's
    RANSAC implementation and I found that very confusing to follow. I read on
    RANSAC for a day, it turns out there are many published works available to
    researchers. For example, in \cite{frahm2006ransac} the authors propose a
    RANSAC method that works with semi-degenerate data. Data degeneracy in
    RANSAC is referred to a situation where the given dataset does not contain
    enough constraints for a unique solution to be obtained via linear models.
    Quaternions are generally more robust to rounding noise and for now, we
    assume degeneracy is impossible. But in our implementation, we deal with
    a similar problem where we have to guess and pick the quaternion solution among
    all possible solutions and I believe the degeneracy test could help us
    with that. Moreover, \cite{shi2013sift} introduces a method for SIFT feature
    point matching with an improved RANSAC algorithm. In their work, they add a
    feedback loop to the random selection feature to improve the selection
    process and shorten the overall execution time. Moreover, I ported in
    SIFT matched features from Matlab to Python and I am in the process of
    comparing outputs line by line. We often forget that some of the
    mathematical formulas we use are not exact and are mere approximations.
    More often than not, where are multiple approximation functions are available
    for mathematical operation. For example, a matrix left division has multiple
    Python implementation and only some match the operation native to Matlab.
    Additionally, I noticed the mean rotation error decreased to half after I
    ported in SIFT matched feature points from Matlab. This points to SIFT's
    higher accuracy in either feature extraction, matching, or both.


    \item DLO Manipulation Milestones: pose estimation and tracking,
    object detection (semantic segmentation), grasping, assembly and
    disassembly, and DLO manipulation.
    \item Pose Estimation (\textcolor{blue}{DLO-01}): On-going under VPQEKF.
    \item Semantic segmentation (\textcolor{blue}{DLO-02}): Per my discussion with Dr. Gans, I
    will explore DL methods for the depth or scale problem.
    \item Grasping Project (\textcolor{blue}{DLO-03}): I am making this a part of the DLO project.
    \item PyTorch Tutorials: Transfer learning.

  \end{itemize}

\section{Intermediate Goals - Fall 2021:}
\begin{itemize}
      \item QEKF: Finish paper.
      \item UR5e: Do the tutorials.
\end{itemize}

\newpage

%Sets the bibliography style to UNSRT and import the
\newpage
\bibliography{ref}
\bibliographystyle{ieeetr}

\end{document}
