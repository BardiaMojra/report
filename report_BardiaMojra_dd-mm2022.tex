\documentclass[11pt]{article}
\usepackage{bookmark}
\usepackage{algorithm}
\usepackage{algpseudocode}
\usepackage{amsfonts}
\usepackage{amsmath}
\usepackage{amssymb}
\usepackage{amsthm}
\usepackage{bm}
\usepackage{color}
\usepackage{comment}
\usepackage{float}
\usepackage{graphicx}
%\usepackage[hidelinks]{hyperref}
\usepackage{makecell}
\usepackage[caption=false,font=footnotesize,subrefformat=parens,labelformat=parens]{subfig}
\usepackage{wrapfig}
\usepackage{url}
\usepackage[table]{xcolor}
\graphicspath{{images/}}
\setlength{\parindent}{0.25in}
\setlength{\parskip}{.05in}
\pagestyle{plain}
%Title, date an author of the document
\title{Progress Report}
\author{Bardia Mojra}


\begin{document}
\maketitle
\thispagestyle{empty}

\bigskip
\bigskip
\begin{center}
 Robotic Vision Lab
\end{center}

\begin{center}
The University of Texas at Arlington
\end{center}

\newpage

\section{Specific Research Goals}
\begin{itemize}
      \item VPQEKF (RAL - April 1st): Work on the paper.
      \item DLO Manipulation Dataset (ICRA - September)
\end{itemize}

\section{To Do}
\begin{itemize}
  \item QEKF Paper - 30\% extension (April 1st):
  \begin{itemize}
      \item Edit VEst section and add updates.
  \end{itemize}
  \item QEKF/QuEst+VEst Implementation (\textcolor{red}{Feb. 28th}):
  \begin{itemize}
      \item Implement QuEst 5-point: On-going - implementing RANSAC.
      \item Feature point extraction: implement semantic segmentation
      \item Implement VEst
      \item Address scale factor (depth-scale) issues: DL solutions?
      \item Address "hand off" issue when objects enter or leave field of view
      \item Real-time streaming images for real-time operation (optional)
      \item Experiments
      \item Noise issue: noise cannot be modeled
  \end{itemize}
  \item  DLO Manipulation:
  \begin{itemize}
      \item Related work literature review
      \item Real dataset + paper (September 2022 - ICRA):
      \begin{itemize}
            \item Watch IROS manipulation workshop videos. - Done.
            \item Design, discuss and build a data collection and test rig.
      \end{itemize}
      \item Unity dataset
      \begin{itemize}
            \item Recreate virtual duplicates of physical test material
            \item Model dynamics and deformity
      \end{itemize}
  \end{itemize}
\end{itemize}


\section{Progress}
The following items are listed in the order of priority:
\begin{itemize}
    \item VPQEKF (RAL - April 1st, 2022): I talked to Dr. Gans and he said we
    are targeting the IEEE Robotics and Automation Letters (RAL) journal for
    this work. Per his recommendation, I only
    implemented \(\phi_3\) metric from \cite{huynh2009metrics} as it is the
    only metric we are interested in. The QuEst algorithm works but it produces
    inaccurate results by a factor of 100-1000 and it seems as
    if suboptimal point correspondences were picked for calculating rotation
    between frames. For this reason, I am now implementing RANSAC similar to
    Kaveh's original implementation. Next, I will implement \emph{semantic
    segmentation} for the KITTI dataset and it will be used to extract more
    consistent point correspondences. It is important to select feature points
    that are only from stationary objects and background; otherwise, including
    point correspondences from moving objects will introduce significant errors
    to pose triangulation.
    \item NBV-Grasping Project: I am making this a part of the DLO project.
    \item DLO Manipulation: No update.
    \item Pose Estimation: I will need it for DLO segment localization.
    \item PyTorch Tutorials: Transfer learning.

  \end{itemize}

\section{Intermediate Goals - Fall 2021:}
\begin{itemize}
      \item QEKF: Finish paper.
      \item UR5e: Do the tutorials.
\end{itemize}

\newpage

%Sets the bibliography style to UNSRT and import the
\newpage
\bibliography{ref}
\bibliographystyle{ieeetr}

\end{document}
