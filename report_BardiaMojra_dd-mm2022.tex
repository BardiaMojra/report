\documentclass[11pt]{article}
\usepackage{bookmark}
\usepackage{algorithm}
\usepackage{algpseudocode}
\usepackage{amsfonts}
\usepackage{amsmath}
\usepackage{amssymb}
\usepackage{amsthm}
\usepackage{bm}
\usepackage{color}
\usepackage{comment}
\usepackage{float}
\usepackage{graphicx}
%\usepackage[hidelinks]{hyperref}
\usepackage{makecell}
\usepackage[caption=false,font=footnotesize,subrefformat=parens,labelformat=parens]{subfig}
\usepackage{wrapfig}
\usepackage{url}
\usepackage[table]{xcolor}
\graphicspath{{images_dd-mm2022/}}
\setlength{\parindent}{0.25in}
\setlength{\parskip}{.05in}
\pagestyle{plain}
%Title, date an author of the document
\title{Progress Report}
\author{Bardia Mojra}


\begin{document}
\maketitle
\thispagestyle{empty}

\bigskip
\bigskip
\begin{center}
 Robotic Vision Lab
\end{center}

\begin{center}
The University of Texas at Arlington
\end{center}

\newpage

\section{Specific Research Goals}
\begin{itemize}
      \item VPQEKF (\textcolor{red}{---}): Work on the paper.
      \item DLO Manipulation Dataset (ICRA - \textcolor{red}{Sept. 1st})
\end{itemize}

\section{To Do}
\begin{itemize}
  \item QEKF Paper - 30\% extension (\textcolor{red}{---}): lack of direction
  \item Implementation (\textcolor{red}{---}):
  \begin{itemize}
      \item Noise issue: noise cannot be modeled - revisit
      \item SfM: RQuEst cannot find solution -- under investigation
  \end{itemize}
  \item  DLO Manipulation: (\textcolor{red}{ICRA - Sept. 1st})
  \begin{itemize}
      \item Work on the paper everyday -- up-coming
      \item ICRA 2022 RL workshops: gym, stable-baseline3, and RL zoo -- on-going
      \item Setup digital twin reinforcement learing setup:
      \begin{itemize}
        \item Unity Robotics extension setup -- on-going.
        \item Design dynamic DLO data collection system.
        \item Build work cell. -- on-going
        \item Collect data and create a dataset.
        \item Define evaluation metrics.
        \item Create a high frequency RGBD dataset with UV-frames and open-loop input control actions as the ground truth.
      \end{itemize}
      \item Real-Time Preception
      \begin{itemize}
        \item Deep learning methods for keypoint pose estimation in real-time.
        \item Use UV dye dataset
        \item Use PVNet-like approach for known-object pose estimation.
      \end{itemize}
      \item Learning DLO Dynamics and System Identification
      \begin{itemize}
            \item List feasible approached for learing DLO dynamics
            \item Model dynamics and deformity in a latent space
      \end{itemize}
      \item Real-Time Control
      \begin{itemize}
        \item Time model inference, using auto-encoders generate the lowest
        dimensional representation for each object.
        \item Use another GAN model for object deformity for each object.
        \item Evaluate encoded representation for accuracy.
        \item Used another GAN to explore other abstraced representations from
        individual encoded representation. In theory, we can create a low
        dimensionsal representation for multiple similar objects, given all
        individual low-dimensional representations. This is inspired by "fundamental
        principles first" approach which has universal applicability.
      \end{itemize}
  \end{itemize}
\end{itemize}

\section{Progress}
The following items are listed in the order of priority:
\begin{itemize}
    \item XEst (\textcolor{red}{RAL ---}): No technical update. Dr. Gans, Asif,
    and I met to discuss Asif's involvement. He didn't seem interested.
    Kaveh's work is incomplete, I shared this Dr. Gans he agreed. I will finish
    it later, I will have to obsess over it. This is where I shine the most.

    \item DLO State Estimation (\textcolor{red}{ICRA - Sept. 1st}): In this
    paper, we will introduce a new augmented dataset for learning DLO local
    dynamics. The dataset is based on a previously available work
    \cite{zhang2021deformable} where we added annotation that allows for more
    efficient learning of the DLO dynamics. Moreover, unlike \cite{zhang2021deformable}
    and \cite{yu2022shape} who used VEA and online-offline Adaptive Control for
    configuration estimation, respectively;
    we deploy \emph{the Koopman Operator} to
    learn the underlying \emph{locally linear dynamics} of the subject DLO.
    \cite{zhang2021deformable} is the main paper I am following as for base
    example. \cite{yu2022shape}, \cite{zhang2021deformable}, and \cite{zhang2021robots}
    are the main papers I am following. Each of those papers have the code
    available for them. \cite{yu2022shape} provides the code for a reinforcement
    learning data collection setup with Unity and UR5. This is great basis for
    my follow up work. \cite{nair2017combining} is the paper I mentioned at the
    meeting. This paper is by Levine's group and I think both dataset and learning
    method are very bad. Dr. Gans agreed. The rope is not dynamic for the most
    part. The test setup does not challenge or interact with object dynamics and
    its configuration is mostly determined by contact friction with the table.
    Moreover, the learing methods is extremely inefficient because most pixels
    in 60K images contain no information regarding object dynamics. My goal is
    to learn dynamics with the Koopman Operator only from regions where we observe
    a bend on the DLO. Everything else is noise in regards to the dynamics.
    I need to read on the Koopman Operator.
    \item Maicol (REU): I want him to continue on ROS2 and Unity and recreate
    \cite{yu2022shape} over the summer. I will help him.
    \item PyTorch Tutorials: Transfer learning.
    \item Omniverse: Apply for access. -- To-Do

  \end{itemize}


\newpage

%Sets the bibliography style to UNSRT and import the
\newpage
\bibliography{ref}
\bibliographystyle{ieeetr}

\end{document}
