\documentclass[11pt]{article}
\usepackage{bookmark}
\usepackage{algorithm}
\usepackage{algpseudocode}
\usepackage{amsfonts}
\usepackage{amsmath}
\usepackage{amssymb}
\usepackage{amsthm}
\usepackage{bm}
\usepackage{color}
\usepackage{comment}
\usepackage{float}
\usepackage{graphicx}
%\usepackage[hidelinks]{hyperref}
\usepackage{makecell}
\usepackage[caption=false,font=footnotesize,subrefformat=parens,labelformat=parens]{subfig}
\usepackage{wrapfig}
\usepackage{url}
\usepackage[table]{xcolor}
\graphicspath{{images/}}
\setlength{\parindent}{0.25in}
\setlength{\parskip}{.05in}
\pagestyle{plain}
%Title, date an author of the document
\title{Progress Report}
\author{Bardia Mojra}


\begin{document}
\maketitle
\thispagestyle{empty}

\bigskip
\bigskip
\begin{center}
 Robotic Vision Lab
\end{center}

\begin{center}
The University of Texas at Arlington
\end{center}

\newpage

\section{Specific Research Goals}
\begin{itemize}
      \item VPQEKF (RAL - April 1st): Work on the paper.
      \item DLO Manipulation Dataset (ICRA - September)
\end{itemize}

\section{To Do}
\begin{itemize}
  \item QEKF Paper - 30\% extension (April 1st):
  \begin{itemize}
      \item Edit VEst section and add updates.
  \end{itemize}
  \item QEKF/QuEst+VEst Implementation (\textcolor{red}{Feb. 28th}):
  \begin{itemize}
      \item Implement QuEst 5-point: Done, debugging.
      \item Feature point extraction: implement semantic segmentation
      \item Implement VEst
      \item Address scale factor (depth-scale) issues: DL solutions?
      \item Address "hand off" issue when objects enter or leave field of view
      \item Real-time streaming images for real-time operation (optional)
      \item Experiments
      \item Noise issue: noise cannot be modeled
  \end{itemize}
  \item  DLO Manipulation:
  \begin{itemize}
      \item Related work literature review
      \item Real dataset + paper (September 2022 - ICRA):
      \begin{itemize}
            \item Design, discuss and build a data collection and test rig.
      \end{itemize}
      \item Unity dataset
      \begin{itemize}
            \item Recreate virtual duplicates of physical test material
            \item Model dynamics and deformity
      \end{itemize}
  \end{itemize}
\end{itemize}


\section{Progress}
The following items are listed in the order of priority:
\begin{itemize}
    \item VPQEKF (\textcolor{red}{RAL - April 1st, 2022}): This week, Cody and
    I went over the C++ implementation by Kaveh. Based on my observations, I
    believe he faced similar issues because there are flags to switch between
    different algebraic decomposition libraries. Dr. Gans and I decided to pivot
    from the Python implementation as it is taking too long to figure out and
    we can not guarantee it will have matching accuracy. I started working on
    the Matlab code this weekend and I am halfway done. I implemented Python
    QEKF is in complete compliance with object-oriented programming (OOP) best
    practices
    which allows for easy and fast source code transfer. In line with that, I
    figured out, implemented, and tested its OOP syntax. The data management
    module is implemented and tested. The QEKF module is implemented but not
    tested and data logging and visualization modules are to be implemented and
    tested in the next few days. I do get distracted, there is no denying that
    and I have always appreciated your guidance. Paying attention to things such
    as OOP is not a distraction rather it is what helps me achieve
    exponential growth in my productivity while maintaining source code
    agility and reliability. This is my understanding, I could be wrong.

    \item DLO Manipulation Milestones: No update other than that Cody asked me
    if I am open to mentoring the EE robotics team this year and helping him with it.
    He suggested the National Robotics Challenge (NRC) but it is not meant for
    Ph.D. students. I will bring up NIST and will ask if they want to help me
    with my next paper/project and in return, I will teach and explain everything
    with will give them credit on the final paper(s). It is important to me to
    build a solid team as well as external alliances. Please advise.


    \item 3D Scanner: It is needed for object manipulation and perception tasks.
    \item Pose Estimation (\textcolor{blue}{DLO-01}): On-going under VPQEKF.
    \item Semantic segmentation (\textcolor{blue}{DLO-02}): Per my discussion with Dr. Gans, I
    will explore DL methods for the depth or scale problem.
    \item Grasping Project (\textcolor{blue}{DLO-03}): I am making this a part of the DLO project.
    \item PyTorch Tutorials: Transfer learning.

  \end{itemize}

\section{Intermediate Goals - Fall 2021:}
\begin{itemize}
      \item QEKF: Finish paper.
      \item UR5e: Do the tutorials.
\end{itemize}

\newpage

%Sets the bibliography style to UNSRT and import the
% \newpage
% \bibliography{ref}
% \bibliographystyle{ieeetr}

\end{document}
