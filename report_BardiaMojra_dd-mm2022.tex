\documentclass[11pt]{article}
\usepackage{bookmark}
\usepackage{algorithm}
\usepackage{algpseudocode}
\usepackage{amsfonts}
\usepackage{amsmath}
\usepackage{amssymb}
\usepackage{amsthm}
\usepackage{bm}
\usepackage{color}
\usepackage{comment}
\usepackage{float}
\usepackage{graphicx}
%\usepackage[hidelinks]{hyperref}
\usepackage{makecell}
\usepackage[caption=false,font=footnotesize,subrefformat=parens,labelformat=parens]{subfig}
\usepackage{wrapfig}
\usepackage{url}
\usepackage[table]{xcolor}
\graphicspath{{images_dd-mm2022/}}
\setlength{\parindent}{0.25in}
\setlength{\parskip}{.05in}
\pagestyle{plain}
%Title, date an author of the document
\title{Progress Report}
\author{Bardia Mojra}


\begin{document}
\maketitle
\thispagestyle{empty}

\bigskip
\bigskip
\begin{center}
 Robotic Vision Lab
\end{center}

\begin{center}
The University of Texas at Arlington
\end{center}

\newpage

\section{Research Plan}
This section outlines my current research plan where the main ideas, target
conference/journal, and expected date of completion for each paper
are provided.
Target conferences: ICRA, IROS (March), CASE (Late Feb.), NIPS.
Target Journals: RAL, CVPR, CORAL.

\begin{itemize}
  \item Koopman-01 (\textcolor{red}{IROS - Dec. 1st - active}):
  Koopman-based MPC control of VTOL-DIP and VTOL-TIP in simulation,
  DLO pose estimation in simulation,
  experiments on choice of basis function and lifting dimensions,
  and performance comparison with optimal, robust, and/or
  adaptive control schemes.\
  \item Koopman-02 (\textcolor{red}{ACC - Sep 30th - active}):
  A review on Koopman-based control schemes. \textcolor{red}{Not enough, make
  it part of another paper.} Read papers and write literature reviews.\

  \item Koopman-03 (\textcolor{black}{RAL - Mar. 1st - status}):
  Extension to Koopman-01, Koopman-based dynamic estimation of DLO,
  collect dynamic DLO dataset,
  prediction of DLO configuration.

  \item Quest-01 (\textcolor{orange}{IROS - Mar. 1st - next}):
  Optimal transform solution for QuEst based on dominant mode decomposition (DMD).
  \item Quest-02 (\textcolor{black}{IROS/RAL - date - status}):
  QuEst-based EKF, structure from motion, and VSLAM, compare performance with
  existing methods.
  \item Koopman-04 (\textcolor{black}{IROS/RAL - date - status}):
  Physics Informed (PI) Koopman-based control of a DLO,
  show obtained is persistant, compare to other non-PI methods, offline-online learning.
  \item Koopman-05 (\textcolor{black}{IROS/RAL - date - status}):
  PI Koopman operator (PIKO) based persistant model for DLOs, low dimensional,
  compare performance, offline-online learing/adapting, fast transfer learning.
  \item Koopman-06 (\textcolor{black}{IROS/RAL - date - status}):
  PIKO-based unit segment model for DLOs, more generalized, should yield better
  performance if number segments are selected online in order to
  obtain optimal representation in real-time given available hardware, compare
  results.
  \item Koopman-07 (\textcolor{black}{IROS/RAL - date - status}):
  DLO dataset, PIKO-based reinforcement learning of real DLO dynamics in a
  digital twin (DT) setting,
  experiments of model persistance, compare learning rate
  with neural network based methods, compare performance with available methods,
  and experiments on learing limitations.
  \item Koopman-08 (\textcolor{black}{IROS/RAL - date - status}):
  Koopman-based real-time control of DLO on GPU.
  \item Koopman-09 (\textcolor{black}{IROS/RAL - date - status}):
  PIKO-based real-time control of DLO on GPU.
  \item Koopman-10 (\textcolor{black}{IROS/RAL - date - status}):
  PIKO-based real-time control of deformable planar objects (DPO).
  \item Koopman-11 (\textcolor{black}{IROS/RAL - date - status}):
  PIKO-based real-time control of deformable volume objects (DVO).
  \item Koopman-12 (\textcolor{black}{IROS/RAL - date - status}):
  PIKO-based unit segment for DPOs, on GPU.
  \item Koopman-13 (\textcolor{black}{IROS/RAL - date - status}):
  PIKO-based unit segment for DVOs, on GPU.
\end{itemize}

\section{To Do}
\begin{itemize}
  \item QEKF Paper (\textcolor{red}{On pause}):
  \begin{itemize}
      \item Noise issue: noise cannot be modeled - DMD is a robust noise on high dimensional orthonormal time series and should be able to denoise QuEst solutions.
      \item SfM: RQuEst cannot find solution - A potential solution is described  briefly above.
  \end{itemize}
  \item  DLO Manipulation: (\textcolor{red}{ICRA - section out of date})
  \begin{itemize}
      \item Setup digital twin reinforcement learing setup:
      \begin{itemize}
        \item Unity Robotics extension setup -- done.
        \item Design dynamic DLO data collection system.
        \item Build work cell. -- done
        \item Collect data and create a dataset.
        \item Define evaluation metrics.
        \item Create a high frequency RGBD dataset with UV-frames and open-loop input control actions as the ground truth.
      \end{itemize}
      \item Real-Time Preception -- on hold
      \item Learning DLO Dynamics and System Identification - PIKO - On-going
  \end{itemize}
\end{itemize}

\newpage
\section{Progress}
The following items are listed in the order of priority:
\begin{itemize}
    \item DLO Manipulation (\textcolor{red}{IROS}):
    This week, I planned on working SimScape multi-body simulation tutorials
    but I did not work on it. I am very much angry and frankly hurt over Dr.
    Beksi choosing to humiliate me in front of the lab. It bothers that he
    provided no valid reason, nor apologized for his conduct. It bothers me
    because I trusted him and listened to him. It bothers me because I have
    been working as hard as anybody else in this lab spite everything I am
    going through. It bothers me because I am doing good research but it is
    dismissed and humiliated because of a conference deadline. I feel alone,
    pushed out and depressed. I was wrong to listen to Dr. Beksi and not going
    on vacations, or going out. Listening to him is making me lose all
    motivation. \\
    \item Maicol (REU): No update, he is busy with classes.\\
    \item DoD SMART (\textcolor{red}{Dec 1st.}): I started the application.\\
    \item XEst (\textcolor{red}{RAL ---}): No update.\\
  \end{itemize}
\newpage

%Sets the bibliography style to UNSRT and import the
\newpage
\bibliography{ref}
\bibliographystyle{ieeetr}

\end{document}
