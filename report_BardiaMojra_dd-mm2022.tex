\documentclass[11pt]{article}
\usepackage{bookmark}
\usepackage{algorithm}
\usepackage{algpseudocode}
\usepackage{amsfonts}
\usepackage{amsmath}
\usepackage{amssymb}
\usepackage{amsthm}
\usepackage{bm}
\usepackage{color}
\usepackage{comment}
\usepackage{float}
\usepackage{graphicx}
%\usepackage[hidelinks]{hyperref}
\usepackage{makecell}
\usepackage[caption=false,font=footnotesize,subrefformat=parens,labelformat=parens]{subfig}
\usepackage{wrapfig}
\usepackage{url}
\usepackage[table]{xcolor}
\graphicspath{{images/}}
\setlength{\parindent}{0.25in}
\setlength{\parskip}{.05in}
\pagestyle{plain}
%Title, date an author of the document
\title{Progress Report}
\author{Bardia Mojra}


\begin{document}
\maketitle
\thispagestyle{empty}

\bigskip
\bigskip
\begin{center}
 Robotic Vision Lab
\end{center}

\begin{center}
The University of Texas at Arlington
\end{center}

\newpage

\section{Specific Research Goals}
\begin{itemize}
      \item VPQEKF (\textcolor{red}{May 30th}): Work on the paper.
      \item DLO Manipulation Dataset (ICRA - \textcolor{red}{Sept. 1st})
\end{itemize}

\section{To Do}
\begin{itemize}
  \item QEKF Paper - 30\% extension (\textcolor{red}{May 30th}):
  \begin{itemize}
      \item Edit VEst section and add updates.
  \end{itemize}
  \item QEKF/QuEst+VEst Implementation (\textcolor{red}{May 30th}):
  \begin{itemize}
      \item OOP Integration: QEKF - Done.
      \item Feature point extraction: implement semantic segmentation
      \item Address scale factor (depth-scale) issues: DL solutions?
      \item Address "hand off" issue when objects enter or leave field of view
      \item Real-time streaming images for real-time operation (optional) - Done
      \item Experiments - Done
      \item Noise issue: noise cannot be modeled - revisit
  \end{itemize}
  \item  DLO Manipulation:  \textcolor{red}{Sept. 1st}
  \begin{itemize}
      \item Find other ICRA dataset papers and summarize the structure. --- On-going.
      \item Dataset (ICRA -  \textcolor{red}{Sept. 1st}):
      \begin{itemize}
            \item Finalize MoCap design, design digital twin work cell. --- Done.
            \item Build work cell.
            \item Collect data and create a dataset.
            \item Create object dynamics ground-truth method, format, and evaluation
            metrics.
      \end{itemize}
      \item Control and Tracking
      \begin{itemize}
            \item Create UR5+DLO simulation in Matlab and begin work on H-Infinity control before Reza leaves for Indiana State.
            \item Model dynamics and deformity
      \end{itemize}
      \item Real-Time Preception
      \begin{itemize}
        \item PVNet approach for known objects
        \item Rope Manipulation dataset
        \item Time model inference, using auto-encoders generate the lowest
        dimensional representation for each object.
        \item Use another GAN model for object deformity for each object.
        \item Evaluate encoded representation for accuracy.
        \item Used another GAN to explore other abstraced representations from
        individual encoded representation. In theory, we can create a low
        dimensionsal representation for multiple similar objects, given all
        individual low-dimensional representations. This is inspired by "fundamental
        principles first" approach which has universal applicability.
      \end{itemize}
      \item
  \end{itemize}
\end{itemize}


\section{Progress}
The following items are listed in the order of priority:
\begin{itemize}
    \item VPQEKF (\textcolor{red}{RAL - April 1st, 2022}): I finished working on
    VPQEKF code and shared results with Dr. Gans. He provided feedback and updated
    the code to better represent the output. But there seem to remain some confusion
    on what the output represent. In short, the system runs 4 QEKF's at the same
    time, one for each pose estimation method. The run-time Kalman filters are
    stored and managed as different instances of the same object, where each instance
    receives pose and velocity estimates corresponding to its assigned pose method.
    For each keyframe, inputs, outputs, and some state variables (e.g. QEKF state
    residual) are stored in a frame buffer. At the end of each frame iteration,
    the frame buffer is passed to the data logger module which stores each variable
    in its corresponding instacne of the log object. Any variable could be logged
    with minimal changes to the code. Dr. Gans and I will discuss the code and its
    output next week. The outputs are attached to this report for your attention.

    \item DLO Dataset: Jerry and Joe seem very busy so I just went with their
    design. It is probably the best design since it take minimal time to build.
    I researched ICRA DLO manipulation papers, made a list and began dissecting
    them systematically. I also started a mind-map for related work section. The
    paper review document and the mind-map are attached for your attention. It
    turns out, I have been over-thinking (as always) the DLO manipulation task.
    It seems that even RGB only dataset could be sufficient to teach an agent to
    manipulation DLOs in real time. The authors in \cite{zhang2021deformable}
    created a state estimation and predition framework very similar to what I
    have thinking. Of course, I did not see it as clearly them but I think it
    provides a good base-line for my paper. Moreover, I can follow their method
    and create a similar dataset with our added contributions such as additional
    objects or sensors (and sensor fusion model).

    \item DLO Control: No update.
    \item DLO Perception: No update. -- might not be needed.

    \item Semantic segmentation (\textcolor{blue}{DLO-02}): Per my discussion with Dr. Gans, I will explore DL methods for the depth or scale problem.
    \item Grasping Project (\textcolor{blue}{DLO-03}): I am making this a part of the DLO project.
    \item PyTorch Tutorials: Transfer learning.

  \end{itemize}

\section{Intermediate Goals - Fall 2021:}
\begin{itemize}
      \item QEKF: Finish paper.
      \item UR5e: Do the tutorials.
\end{itemize}

\newpage

%Sets the bibliography style to UNSRT and import the
\newpage
\bibliography{ref}
\bibliographystyle{ieeetr}

\end{document}
