\documentclass[11pt]{article}
\usepackage{bookmark}
\usepackage{algorithm}
\usepackage{algpseudocode}
\usepackage{amsfonts}
\usepackage{amsmath}
\usepackage{amssymb}
\usepackage{amsthm}
\usepackage{bm}
\usepackage{color}
\usepackage{comment}
\usepackage{float}
\usepackage{graphicx}
%\usepackage[hidelinks]{hyperref}
\usepackage{makecell}
\usepackage[caption=false,font=footnotesize,subrefformat=parens,labelformat=parens]{subfig}
\usepackage{wrapfig}
\usepackage{url}
\usepackage[table]{xcolor}
\graphicspath{{images_dd-mm2022/}}
\setlength{\parindent}{0.25in}
\setlength{\parskip}{.05in}
\pagestyle{plain}
%Title, date an author of the document
\title{Progress Report}
\author{Bardia Mojra}


\begin{document}
\maketitle
\thispagestyle{empty}

\bigskip
\bigskip
\begin{center}
 Robotic Vision Lab
\end{center}

\begin{center}
The University of Texas at Arlington
\end{center}

\newpage

\section{Specific Research Goals}
\begin{itemize}
      \item VPQEKF (\textcolor{red}{---}): Work on the paper.
      \item DLO Manipulation Dataset (ICRA - \textcolor{red}{Sept. 1st})
\end{itemize}

\section{To Do}
\begin{itemize}
  \item QEKF Paper - 30\% extension (\textcolor{red}{---}): lack of direction
  \item Implementation (\textcolor{red}{---}):
  \begin{itemize}
      \item Noise issue: noise cannot be modeled - revisit
      \item SfM: RQuEst cannot find solution -- under investigation
  \end{itemize}
  \item  DLO Manipulation: (\textcolor{red}{ICRA - Sept. 1st})
  \begin{itemize}
      \item Work on the paper everyday -- up-coming
      \item ICRA 2022 RL workshops: gym, stable-baseline3, and RL zoo -- on-going
      \item Setup digital twin reinforcement learing setup:
      \begin{itemize}
        \item Unity Robotics extension setup -- on-going.
        \item Design dynamic DLO data collection system.
        \item Build work cell. -- on-going
        \item Collect data and create a dataset.
        \item Define evaluation metrics.
        \item Create a high frequency RGBD dataset with UV-frames and open-loop input control actions as the ground truth.
      \end{itemize}
      \item Real-Time Preception
      \begin{itemize}
        \item Deep learning methods for keypoint pose estimation in real-time.
        \item Use UV dye dataset
        \item Use PVNet-like approach for known-object pose estimation.
      \end{itemize}
      \item Learning DLO Dynamics and System Identification
      \begin{itemize}
            \item List feasible approached for learing DLO dynamics
            \item Model dynamics and deformity in a latent space
      \end{itemize}
      \item Real-Time Control
      \begin{itemize}
        \item Time model inference, using auto-encoders generate the lowest
        dimensional representation for each object.
        \item Use another GAN model for object deformity for each object.
        \item Evaluate encoded representation for accuracy.
        \item Used another GAN to explore other abstraced representations from
        individual encoded representation. In theory, we can create a low
        dimensionsal representation for multiple similar objects, given all
        individual low-dimensional representations. This is inspired by "fundamental
        principles first" approach which has universal applicability.
      \end{itemize}
  \end{itemize}
\end{itemize}

\section{Progress}
The following items are listed in the order of priority:
\begin{itemize}
    \item XEst (\textcolor{red}{RAL ---}): I printed out saved the outputs for
    5-point algorithm \cite{nister2004efficient} as well as QuEst as they
    estimate the essential matrix and
    quaternion solutions respective. We need to investigate why QuEst can not
    find a realizable solution. Although Quaternions are very robust to noise,
    they very sensitive to noise when estimating them from point correspondences.
    Coming up with a solution could take some time, so I put the project on
    pause for now.
    \item DLO Dataset (\textcolor{red}{ICRA - Sept. 1st}): I am working on
    finishing Unity tutorials today. My notes are attached. Tomorrow, Maicol and
    I will work Unity-ROS integration with UR5. Then, I will recreate the workcell.
    \item Maicol (REU): He will do what he wants, but I think he is smart
    enough not to do that. After your comment, I do not
    feel comfortable assigning tasks to Maicol. Moveover, I explained to him
    that this is your lab and you set the rules. If he is interested to
    volunteer/work at UTARI under my project, I will have to assume
    responsibility for him. That is what I told Dr. Gans that he agreed to
    him volunteering in Spring. I explained to him, I have no say on pay and
    Dr. Gans and Dr. Clements decide on that. I explained, his skillset,
    contribution and performance are the defining factors for a paid position.
    I explained to him that this is about mutual interested (him learning, and
    me getting more done with my research) and I offered him a much fairer deal
    than when I received as an undergrad. I like him but I don't need his help,
    to be clear.
    \item PyTorch Tutorials: Transfer learning.
    \item Omniverse: Apply for access. -- To-Do

  \end{itemize}


\newpage

%Sets the bibliography style to UNSRT and import the
\newpage
\bibliography{ref}
\bibliographystyle{ieeetr}

\end{document}
